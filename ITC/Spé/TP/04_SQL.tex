%-------------------------------------------------------------------------------
%-------------------------------------------------------------------------------
%-------------------------------------------------------------------------------
\chapter{SQL : révisions}
%-------------------------------------------------------------------------------
%-------------------------------------------------------------------------------
\thispagestyle{empty}
%-------------------------------------------------------------------------------
%--------------------------------------------------------------------------
%--------------------------------------------------------------------------
\section{Rappels}
%--------------------------------------------------------------------------
%--------------------------------------------------------------------------
\subsection{Structure}
%--------------------------------------------------------------------------
%--------------------------------------------------------------------------
Les bases de données relationnelles utilisent des données regroupées en {\bf tables} (ou {\bf relations}). 

\begin{itemize}
    \item Chaque ligne d'une relation représente une {\bf entité} (personne, objet, action, relation, \dots).
    \item Une entité est un $n$-uplet (tuple) de valeurs.
    \item L'ensemble des entités est l'{\bf extension} de la relation.
    \item Chaque élément d'un $n$-uplet est un {\bf attribut} de l'entité.
    \item L'ensemble des attributs (les colonnes) est le {\bf schéma} de la relation.
    \item Chaque attribut prend des valeurs dans un ensemble défini : son {\bf domaine}.
    \item Une {\bf clé primaire} est un ensemble d'attributs qui identifie chaque entité de la relation de manière unique et minimale.
    \item Souvent, la clé primaire sera un attribut dédié, un identifiant.
\end{itemize}
%--------------------------------------------------------------------------
%--------------------------------------------------------------------------
\subsection{Requêtes SQL}
%--------------------------------------------------------------------------
%--------------------------------------------------------------------------
On interroge les bases de données à l'aide d'un langage standardisé : SQL.

Une requête produit un résultat qui est lui même une table (ou une valeur unique), il sera donc possible de composer ou combiner les requêtes.

L'ordre des instructions dans une requête est fixé :

\begin{lstlisting}[language=SQL]
SELECT ...
FROM ...
(WHERE ...)
(GROUP BY ...)
(HAVING ...)
(ORDER BY ...)
(LIMIT ...)
(OFFSET ...)
\end{lstlisting}

Seules les deux premières instructions sont obligatoires. Les instructions à partir de \type{from} sont exécutées dans l'ordre, \type{select} chapeaute la requête.

\newpage

\begin{description}
    \item[Select] est suivi de l'énumération des valeurs à afficher séparées par des virgules.
    \begin{itemize}
        \item Ce peut être des attributs de la tables, {\bf *} si on les veut tous.
        \item Ce peut être des valeurs calculées à partir des attributs, il sera alors utile de donner un nom (un alias) au résultat : {\it calcul} {\tt as} {\it nom}.
        \item Ce peut être le résultat d'un calcul statistique sur les lignes sélectionnées, regroupées ou non (voir {\bf Group by}). On peut utiliser la somme des valeurs d'un attribut \type{sum({\it a})}, la moyenne \type{avg({\it a})}, le maximum \type{max({\it a})}, le minimum \type{min({\it a})}, le nombre d'entités sélectionnées \type{count()}.
        \item Pour éviter les répétitions on peut utiliser {\tt select distinct}.

    \end{itemize}
    \item[From] est suivi de la table utilisée. Ce peut être
       \begin{itemize}
        \item une table simple de la relation,
        \item un produit de tables, on les énumère avec des virgules pour séparer,
        \item une {\bf jointure} de tables.
    \end{itemize}
    \item[Where] est suivi des conditions que doivent vérifier les attributs (ou les valeurs calculées en utilisant leur alias). On peut les combiner avec \type{and}, \type{or} et \type{not}.
    \item[Group by] est suivi d'un attribut : il partitionne la table selon les valeurs de cet attribut et permet un calcul statistique par paquet. Quand cette instruction est utilisée on ne peut utiliser dans \type{select} que l'attribut du regroupement\footnote{On peut aussi utiliser des attributs qui ont la même valeur par paquet.} et des valeurs statistiques.
    \item[Having] n'est utile qu'avec une instruction \type{Group by}. Il permet de donner des conditions portant sur les valeurs statistiques calculées.
    \item[Order by] suivi d'un nom d'attribut ou d'alias permet de trier par ordre croissant selon les valeurs de cet attribut. On peut faire suivre par \type{desc} si on veut un ordre décroissant.
    \item[Limit] n'est utile qu'avec une instruction \type{Order by}. Il permet de limiter le nombre de résultats affichés
    \item[Offset] n'est utile qu'avec une instruction \type{Limit}. Il permet de décaler les instructions. \type{Limit 3 Offset 4} affiche les résultats des rangs 4 à 6.
\end{description}
%--------------------------------------------------------------------------
%--------------------------------------------------------------------------
\subsection{Jointures}
%--------------------------------------------------------------------------
%--------------------------------------------------------------------------
{\bf Remarque} : quand on utilise plusieurs tables, on accède aux attributs en donnant aussi la table d'où ils proviennent : {\it table.nom}. 

Il sera souvent utile de donner un alias court aux tables : \type{From} {\it table} \type{as} {\it t}.

\medskip

Quand on utilise plusieurs table, il y a souvent des liens entre celles-ci : un attribut d'une table réfère à un attribut d'une autre table. Cela sera parfois indiqué dans une représentation des schémas.

%-------------------------------------------------------------------------------
\begin{center}
\tikzstyle{table}=[draw,shape=rectangle,text width=28mm,align=center,minimum height=6mm]
\tikzpicture
\node[table] at (-1, 0.0) {\bf PATIENT};
\node[table] at (-1,-0.6) (id-p) {id};
\node[table] at (-1,-1.2) {Nom};
\node[table] at (-1,-1.8) {Prenom};
\node[table] at (-1,-2.4) {Adresse};
\node[table] at (-1,-3.0) {email};
\node[table] at (-1,-3.6) {naissance};
\node[table] at (4, 0.0) {\bf MEDICAL};
\node[table] at (4,-0.6) {id};
\node[table] at (4,-1.2) {data1};
\node[table] at (4,-1.8) {data2};
\node[table] at (4,-2.4) {data3};
\node[table] at (4,-3.0) (id-m) {id\_patient};
\node[table] at (4,-3.6) {etat};
\draw[thick, <->] (id-p) -- +(2.5, 0) |-  (id-m);
\endtikzpicture
\end{center}
%--------------------------------------------------------------------------

Il est recommandé d'utiliser cette relation en améliorant le produit par une jointure. Dans l'exemple cela donnerait 
\begin{lstlisting}[language=SQL]
FROM patient as p Join medical as m On p.id = m.id_patient
\end{lstlisting}
Le nom d'un patient sera alors connu par \type{m.nom}
%--------------------------------------------------------------------------
\newpage
%--------------------------------------------------------------------------
\subsection{Assemblages de requêtes}
%--------------------------------------------------------------------------
%--------------------------------------------------------------------------
On peut combiner des requêtes de plusieurs manières.

\begin{enumerate}
    \item Quand deux requêtes produisent des tables avec le même schéma, on peut en faire l'union : \type{union}, l'intersection : \type{intersection} ou la différence : \type{except}. Le mot clé est simplement placé entre les deux requêtes.
    \item Une requête peut produire une valeur unique (avec une fonction statistique), cette valeur peut être utilisée dans une condition (\type{where} ou \type{having}). On ne peut malheureusement pas sauvegarder la valeur dans une variable, il faut l'utiliser en écrivant la requête entre parenthèses.
    \item En général une requête produit une nouvelle table. On peut alors effectuer une requête sur cette table. Ici encore on écrit la requête entre parenthèses dans l'instruction \type{From}.
\end{enumerate}
%--------------------------------------------------------------------------
%--------------------------------------------------------------------------
\section{Présentation}
%--------------------------------------------------------------------------
%--------------------------------------------------------------------------

{\it Nous allons utiliser une base de données qui contient de nombreuses tables.

Certaines sont des tables qui contiennent des descriptions, par exemple \textbf{Pays}.

D'autres représentent des associations entre des données, par exemple \textbf{Frontiere}.

\medskip

Voici les tables tables utilisées et leurs attributs. La clé primaire est soulignée.

\begin{itemize}
\item {\bf Appartenance} : \underline{Pays, Continent}, Pourcentage
\item {\bf Continent} : \underline{Nom}, Superficie
\item {\bf Frontiere} : \underline{Pays1, Pays2}, Longueur
\item {\bf Ile} : \underline{Nom}, Iles, Superficie, Altitude, Type, Longitude, Latitude 
\item {\bf IleDans} : \underline{Ile, Mer, Lac, Riviere}
\item {\bf IlePays} : \underline{Ile, Pays}, Province
\item {\bf Langage} : \underline{Pays, Nom}, Percentage
\item {\bf MembreOrganisation} : \underline{Organisation, Pays}, Type
\item {\bf Montagne} : \underline{Nom}, Chaine, Altitude, Type, Longitude, Latitude 
\item {\bf MontagnePays} : \underline{Montagne, Pays, Province}. 
\item {\bf Organisation} : \underline{Abbreviation}, Nom, Ville, \dots
\item {\bf Pays} : Nom, \underline{Code}, Capitale, Province, Superficie, Population
\item {\bf Politique} : \underline{Pays}, DateIndependance, AncienneDependance, Dependance, \dots
\item {\bf Riviere} : \underline{Nom}, Riviere, Lac, Mer, \dots
\item {\bf RivierePays} : \underline{Riviere, Pays, Province}
\end{itemize}

\medskip

Dans les tables d'associations de nom du pays est le code de la table {\bf Pays}.

Dans chaque question on a donné le nom de la table ou des tables utilisées.

La table n'est pas complète et comporte des erreurs.}
%--------------------------------------------------------------------------
%--------------------------------------------------------------------------
\section{Requêtes sur une seule table}
%--------------------------------------------------------------------------
%--------------------------------------------------------------------------
\begin{Exercise}
Déterminer la liste des continents et de leur superficie, par ordre croissant.
\end{Exercise}

{\bf Continent} 

%--------------------------------------------------------------------------
\begin{Answer}
\begin{lstlisting}[language=SQL]
select *
from continent
order by Superficie
\end{lstlisting}
\end{Answer}
%--------------------------------------------------------------------------
\begin{Exercise}
Quels sont les pays de plus de $10^8$ habitants ? Il y en a 10.
\end{Exercise}

{\bf Pays}

%--------------------------------------------------------------------------
\begin{Answer}
\begin{lstlisting}[language=SQL]
select nom, population
from Pays
where population > 100000000
order by 2 desc
\end{lstlisting}
\end{Answer}
%--------------------------------------------------------------------------
\begin{Exercise}
Quels sont les volcans ({\tt type = "volcano"}) de plus de 6000 m ? Il y en a 7.
\end{Exercise}

{\bf Montagne}

%--------------------------------------------------------------------------
\begin{Answer}
\begin{lstlisting}[language=SQL]
select nom, altitude
from Montagne 
where type = 'volcano'  AND altitude > 6000
\end{lstlisting}
\end{Answer}
%--------------------------------------------------------------------------
\newpage
{\it Une rivière se jette dans une autre rivière (elle est un affluent), dans un lac ou dans une mer. C'est ce qu'indiquent les attributs {\it Rivière}, {\it Lac} ou {\it Mer}. En théorie un seul de ces attributs n'est pas vide mais une rivière peut aussi passer dans un lac avant d'atteindre une autre rivière ou une mer.}
%--------------------------------------------------------------------------
\begin{Exercise}
Quels sont les affluents du Rhin ({\bf Rhein}) ? Il y en a 4.
\end{Exercise}
{\bf Riviere} 

%--------------------------------------------------------------------------
\begin{Answer}
\begin{lstlisting}[language=SQL]
select nom
from Riviere
where riviere = "Rhein"
\end{lstlisting}
\end{Answer}
%--------------------------------------------------------------------------
%--------------------------------------------------------------------------
\section{Fonctions d’agrégation}
%--------------------------------------------------------------------------
%--------------------------------------------------------------------------
\begin{Exercise}
Quelle est la longueur de la frontière (terrestre) de la France (code "F") ? (2892.4 km)
\end{Exercise}
{\bf Frontiere} 

%--------------------------------------------------------------------------
\begin{Answer}
\begin{lstlisting}[language=SQL]
select sum(longueur)
from Frontiere 
where Pays1 = "F" or Pays2 = "F"
\end{lstlisting}
\end{Answer}
%--------------------------------------------------------------------------
\begin{Exercise}
Quelles sont les îles faisant partie des Caraïbes ({\tt iles = "Caraibes"}) avec leur superficie ?
\end{Exercise}

{\bf Ile} 
%--------------------------------------------------------------------------
\begin{Answer}
\begin{lstlisting}[language=SQL]
select nom,superficie
from Ile
where Iles = "Caraibes"
order by 1
\end{lstlisting}
\end{Answer}
%--------------------------------------------------------------------------
\begin{Exercise}
Combien y a-t-il d'îles dans les Caraïbes ? \texttt{count} (22)

Quelle est la superficie totale ? (12165 km${}^2$)

Quelle est la moyenne des points culminants ? (825 m)
\end{Exercise}

{\bf Ile} 
%--------------------------------------------------------------------------
\begin{Answer}
\begin{lstlisting}[language=SQL]
select count() as nombre, sum(Superficie) as surface, avg(Altitude) as altitude 
from ile
where iles = "Caraibes"
\end{lstlisting}
\end{Answer}
%--------------------------------------------------------------------------
\begin{Exercise}
Quelles sont les mers dans lesquelles se jettent plus 5 rivières ou plus (\texttt{having}) ? Il y en a 5.

On excluera les rivières qui ne sont pas des fleuves par la condition : \texttt{mer is not null}.
\end{Exercise}
{\bf Riviere}
%--------------------------------------------------------------------------
\begin{Answer}
\begin{lstlisting}[language=SQL]
select count() as nombre, mer
from Riviere
where mer is not null
group by mer
having nombre >4
\end{lstlisting}
\end{Answer}
%--------------------------------------------------------------------------
\begin{Exercise}
Quelles sont les villes qui sont le siège de plus de 5 organisations ?
Il y en a 6.
\end{Exercise}
{\bf Organisation}
%--------------------------------------------------------------------------
\begin{Answer}
\begin{lstlisting}[language=SQL]
select Ville, count() as nbSieges 
from Organisation
where Ville is not Null
group by Ville
having nbSieges >= 5
\end{lstlisting}
\end{Answer}
%--------------------------------------------------------------------------
\begin{Exercise}
Quelles sont les langues parlées dans 5 pays au moins par plus de 25 \% de la population ?

Il y en a 4.
\end{Exercise}
{\bf Langage}
%--------------------------------------------------------------------------
\begin{Answer}
\begin{lstlisting}[language=SQL]
select nom, count() as nb
from langage
where percentage >25
group by nom
having nb >4\end{lstlisting}
\end{Answer}
%--------------------------------------------------------------------------
%--------------------------------------------------------------------------
\section{Jointures}
%--------------------------------------------------------------------------
%--------------------------------------------------------------------------
\begin{Exercise}
Quels sont les pays membres de l'UNESCO ? Il y en a 192 (188 vrais membres).
\end{Exercise}
{\bf MembreOrganisation}, {\bf Pays}
%--------------------------------------------------------------------------
\begin{Answer}
\begin{lstlisting}[language=SQL]
select p.nom
from Pays as p join MembreOrganisation as mo on p.Code = mo.Pays
where mo.Organisation = "UNESCO"
\end{lstlisting}
\end{Answer}
%--------------------------------------------------------------------------
\begin{Exercise}
Quels sont les pays qui ont été des colonies de la France (\texttt{AncienneDependance = "F"}) ? (24)

\end{Exercise}

{\bf Politique}, {\bf Pays}
%--------------------------------------------------------------------------
\begin{Answer}
\begin{lstlisting}[language=SQL]
select p.nom
from Pays as p join Politique as pol on p.code = pol.Pays
where pol.AncienneDependance = "F"\end{lstlisting}
\end{Answer}
%--------------------------------------------------------------------------
\begin{Exercise}
Quels sont les pays qui dans lesquels l'anglais est la langue pour plus de 10 \% de la population ?

\end{Exercise}
{\bf Langage}, {\bf Pays}
%--------------------------------------------------------------------------
\begin{Answer}
\begin{lstlisting}[language=SQL]
select p.nom
from langage as l join pays as p on p.code = l.pays
where l.percentage > 10 and l.nom = "English"\end{lstlisting}
\end{Answer}
%--------------------------------------------------------------------------
\newpage
%--------------------------------------------------------------------------
\begin{Exercise}
Quels sont les pays qui sont traversés par le Danube ({\bf Donau}) ? Il y en a 10.
\end{Exercise}
{\bf RivierePays}, {\bf Pays}
%--------------------------------------------------------------------------
\begin{Answer}
\begin{lstlisting}[language=SQL]
select distinct p.nom
from Pays as p join RivierePays as r on p.code = r.Pays
where r.Riviere = "Donau"
\end{lstlisting}
\end{Answer}
%--------------------------------------------------------------------------
\begin{Exercise}
Quelles sont les montagnes d'Amérique de plus de 6000 m avec le nom du pays dans lequel elles sont situées et leur altitude ? Il y en a 16.
\end{Exercise}
{\bf MontagnePays}, {\bf Appartenance}, {\bf Pays}, {\bf Montagne}
%--------------------------------------------------------------------------
\begin{Answer}
\begin{lstlisting}[language=SQL]
select distinct m.Nom, p.Nom, m.Altitude
from Appartenance as a join MontagnePays as mp on a.Pays = mp.Pays
                       join Montagne as m on m.Nom = mp.Montagne
                       join Pays as p on p.Code = a.Pays
where m.Altitude > 6000 and a.Continent = "America"
\end{lstlisting}

\newpage
\end{Answer}
%--------------------------------------------------------------------------
\begin{Exercise}
Quels sont les affluents des affluents du Rhin ? Il y en a 3.
\end{Exercise}

{\bf Riviere} 2 fois
%--------------------------------------------------------------------------
\begin{Answer}
\begin{lstlisting}[language=SQL]
select r.nom
from Riviere as r join (select nom, riviere from Riviere) as aff 
                  on r.riviere = aff.nom
where aff.riviere = "Rhein"
\end{lstlisting}

On peut aussi composer les requêtes avec {\bf in}.

\begin{lstlisting}[language=SQL]
select nom
from Riviere 
where riviere in (select nom
                  from Riviere
                  where riviere = "Rhein")
\end{lstlisting}
\end{Answer}
%--------------------------------------------------------------------------
\begin{Exercise}
Quels sont les pays dont la plus grande partie est en Europe ? Il y en a 51.
\end{Exercise}

{\bf Pays}, {\bf Appartenance}
%--------------------------------------------------------------------------
\begin{Answer}
\begin{lstlisting}[language=SQL]
select p.Nom 
from Pays as p join Appartenance as a on a.Pays = p.code
where a.Continent = "Europe" and a.Pourcentage > 50
\end{lstlisting}
\end{Answer}
%--------------------------------------------------------------------------
\begin{Exercise}
Quels sont les organisations qui rassemblent des pays avec un total de plus de $5.10^9$ habitants ?

Il y en a 27.
\end{Exercise}
{\bf Organisation}, {\bf Pays}, {\bf MembreOrganisation}
%--------------------------------------------------------------------------
\begin{Answer}
\begin{lstlisting}[language=SQL]
select o.Nom, sum(p.Population) as Pop 
from Organisation as o join MembreOrganisation 
                            as mo on o.Abbreviation=mo.Organisation
                       join Pays as p on p.Code = mo.Pays
group by mo.Organisation
having pop > 5000000000
\end{lstlisting}
\end{Answer}
%--------------------------------------------------------------------------
\begin{Exercise}
Dans quels pays existe-t-il une île dans un Lac ? Il y en a 4 (2 îles au Canada).
\end{Exercise}
{\bf IleDans}, {\bf IlePays}, {\bf Pays}
%--------------------------------------------------------------------------
\begin{Answer}
\begin{lstlisting}[language=SQL]
select p.Nom, i.Ile as Ile, i.Lac 
from Pays as p Join IlePays as ip on p.code=ip.Pays
               Join IleDans as i on ip.Ile=i.Ile
where i.Lac is not Null
\end{lstlisting}
\end{Answer}
%--------------------------------------------------------------------------
\begin{Exercise}
Quels sont les pays qui ont une montagne dans la chaîne des Alpes ({\bf "Alps"}) ? Il y en a 5.
\end{Exercise}

{\bf Pays}, {\bf MontagnePays}, {\bf Montagne}
%--------------------------------------------------------------------------
\begin{Answer}
\begin{lstlisting}[language=SQL]
select distinct p.Nom
from Pays as p join MontagnePays as mp on p.Code = mp.Pays
               join Montagne as m on m.Nom = mp.Montagne
where m.Chaine = "Alps"
\end{lstlisting}
\end{Answer}
%--------------------------------------------------------------------------
\begin{Exercise}
Quels sont les fleuves qui passent en France ?
\end{Exercise}

{\bf Riviere}, {\bf RivierePays}, {\bf Pays}
%--------------------------------------------------------------------------
\begin{Answer}
\begin{lstlisting}[language=SQL]
select distinct r.nom
from Riviere as r join RivierePays as rp on r.nom = rp.riviere
                               join Pays as p on rp.pays = p.code
where p.nom ="France" and r.mer is not null
\end{lstlisting}
\newpage
\end{Answer}
%--------------------------------------------------------------------------
%--------------------------------------------------------------------------
\section{Sous-requêtes et combinaisons}
%--------------------------------------------------------------------------
%--------------------------------------------------------------------------
\begin{Exercise}
Quels sont les pays qui font partie de l'ONU ({\bf UN}) mais pas de l'UNESCO ? Il y en a 6.
\end{Exercise}
{\bf Pays}, {\bf MembreOrganisation}
%--------------------------------------------------------------------------
\begin{Answer}
\begin{lstlisting}[language=SQL]
select p.Nom
from Pays as p join MembreOrganisation as mo on p.Code = mo.Pays
where mo.Organisation = "UN" AND type = "member"

except

select p.Nom
from Pays as p join MembreOrganisation as mo on p.Code = mo.Pays
where mo.Organisation = "UNESCO" AND type = "member"
\end{lstlisting}
\end{Answer}
%--------------------------------------------------------------------------
\begin{Exercise}
Quels sont les pays qui ont une frontière avec la Russie ({\bf Russia}, code {\bf R}) ? Il y en a 14.
\end{Exercise}
{\bf Frontiere}, {\bf Pays}
%--------------------------------------------------------------------------
\begin{Answer}
\begin{lstlisting}[language=SQL]
select p.Nom
from Pays as p join Frontiere as f on p.Code = f.Pays1
where f.Pays2 = "R"

union

select p.Nom
from Pays as p join Frontiere as f on p.Code = f.Pays2
where f.Pays1 = "R"
\end{lstlisting}
\end{Answer}
%--------------------------------------------------------------------------
\begin{Exercise}
Combien ont d'affluents les fleuves qui passent en France ? (Seuls 3 ont des affluents dans la base)
\end{Exercise}

{\bf Riviere} deux fois, {\bf RivierePays}, {\bf Pays}

La réponse n'est pas que la Seine a quatre affluents dans la base.
%--------------------------------------------------------------------------
\begin{Answer}
La base \type{RivierePays} indique les rivière plusieurs fois, pour chaque région parcourue. Il faut donc ne prendre que les fleuves distincts.
\begin{lstlisting}[language=SQL]
select f.fleuve, count()
from (select distinct r.nom as fleuve
          from Riviere as r join RivierePays as rp 
                                on r.nom = rp.riviere
                            join Pays as p 
                                on rp.pays = p.code
          where p.nom ="France" and r.mer is not null) as f
		  join Riviere as r1 on f.fleuve = r1. riviere
group by f.fleuve
\end{lstlisting}
\end{Answer}
%--------------------------------------------------------------------------
\begin{Exercise}
Quels sont les langues parlées par plus de 100.000.000 locuteurs ?
(7)
\end{Exercise}

{\bf Pays}, {\bf Langage}
%--------------------------------------------------------------------------
\begin{Answer}
\begin{lstlisting}[language=SQL]
select langue, sum(locuteurs) as total
from (
  select p.nom, l.nom as langue, p.population*l.percentage/100 as locuteurs
  from langage as l join pays as p on p.code = l.pays)
  group by langue
  having total > 100000000
  order by total desc

\end{lstlisting}
\newpage
\end{Answer}
%--------------------------------------------------------------------------
\begin{Exercise}
Quels sont les pays qui partagent une même montagne ? Il y en a 69.
\end{Exercise}

{\bf Pays}, {\bf MontagnePays}, {\bf Montagne}
%--------------------------------------------------------------------------
\begin{Answer}
\begin{lstlisting}[language=SQL]
select distinct mm.Nom as NomMontagne, p.Nom as NomPays
from (select Montagne as Nom, count() as nb
      from (select distinct Montagne,Pays 
            from MontagnePays)
            group by Nom) as mm
           join MontagnePays as mp on  mp.Montagne = mm.Nom
           join Pays as p on p.Code = mp.Pays
where mm.nb > 1
order by 1
\end{lstlisting}
\end{Answer}
%--------------------------------------------------------------------------
\begin{Exercise}
Quels sont les pays d'Europe qui ont une montagne dans la base ? On donnera aussi le nom et l'altitude de la plus haute montagne. Il y en a 20.
\end{Exercise}

{\bf Pays}, {\bf MontagnePays}, {\bf Montagne}, {\bf Appartenance}
%--------------------------------------------------------------------------
\begin{Answer}
\begin{lstlisting}[language=SQL]
select p.Nom, mp.Montagne, max(m.Altitude)
from Pays as p join Appartenance as a on a.Pays = p.code
               join MontagnePays as mp on p.Code = mp.Pays
               join Montagne as m on m.Nom = mp.Montagne
where a.Continent = "Europe" and a.Pourcentage > 50
group by mp.Pays
\end{lstlisting}
\end{Answer}
%--------------------------------------------------------------------------
% %--------------------------------------------------------------------------
% \tikzstyle{mot}=[draw,
%                  shape=rectangle,
%                  rounded corners = 3mm,
%                  align=center,
%                  inner sep =2mm,
%                  thick,
%                  minimum height=6mm,
%                  minimum width=6mm,
%                 ]
% %--------------------------------------------------------------------------
% \tikzstyle{gvr}=[draw,
%                  shape=rectangle,
%                  inner sep = 1mm,
%                  align=center,
%                  minimum height=6mm,
%                  minimum width=18mm,
%                 ]
% %--------------------------------------------------------------------------
% \tikzstyle{var}=[draw,
%                  shape=rectangle,
%                  inner sep = 1mm,
%                  minimum height=6mm,
%                  align=center,
%                 ]
% %--------------------------------------------------------------------------
% \tikzstyle{fle}=[->,
%                  rounded corners = 1.5mm,
%                  thick,
%                 ]
% %--------------------------------------------------------------------------
% \tikzstyle{drt}=[rounded corners = 1.5mm,
%                  thick,
%                 ]
% %--------------------------------------------------------------------------
% \tikzstyle{gau}=[<-,
%                  rounded corners = 1.5mm,
%                  thick,
%                 ]
% %--------------------------------------------------------------------------
% \def\dd{3mm}
% \def\h{3mm}
% \def\hv{4mm}
% %--------------------------------------------------------------------------
% \tikzpicture
% %--------------------------------------------------------------------------
% \coordinate(dl1) at (0,0)       {};
% \node[mot] (sel) at ($(dl1)+(1.3,0)$)     {\bf SELECT};
% \node[mot] (dct) at ($(sel)+(2.8,-2*\h)$) {\bf DISTINCT};
% \node[gvr] (cl1) at ($(dct)+(3.4,2*\h)$)  {\tt colonne};
% \node[gvr] (ccl) at ($(cl1)+(0,2*\hv)$)   {\tt calcul};
% \node[gvr] (Fag) at ($(cl1)+(0,-2*\hv)$)  {\tt fn\_agreg};
% \node[mot] (as1) at ($(cl1)+(3,-2*\h)$)   {\bf AS};
% \node[var] (nm1) at ($(as1)+(1.2,0)$)       {\tt alias};
% \node[mot] (vgl) at ($(cl1)+(2,4*\hv)$)   {\bf ,};
% \coordinate(fl1) at ($(dl1)+(13.5,0)$)       {};
% %--------------------------------------------------------------------------
% \draw[fle]($(dl1)+(-4*\dd,0)$)-- (sel.west);
% \draw[fle](sel.east) -- (cl1.west);
% \draw[drt](cl1.east) -- (fl1);
% \draw[gau](dct.west) -| +(-\dd,\h)  |- +(-2*\dd,2*\h);
% \draw[drt](dct.east) -| +(\dd,\h)   |- +(2*\dd,2*\h);
% \draw[gau](ccl.west) -| +(-\dd,-\hv)|- +(-2*\dd,-2*\hv);
% \draw[fle](ccl.east) -| +(\dd,-\hv) |- +(2*\dd,-2*\hv);
% \draw[gau](Fag.west) -| +(-\dd,\hv) |- +(-2*\dd,2*\hv);
% \draw[drt](Fag.east) -| +(\dd,\hv)  |- +(2*\dd,2*\hv);
% \draw[gau](as1.west) -| +(-\dd,\h)  |- +(-2*\dd,2*\h);
% \draw[fle](as1.east) -- (nm1.west);
% \draw[fle](nm1.east) -| +(\dd,\h)   |- +(2*\dd,2*\h);
% \draw[fle] ($(nm1.east)+(2*\dd,2*\h)$) -| +(\dd,2*\hv) |- (vgl.east);
% \draw[fle] (vgl.west) -| ($(dct.east)+(\dd,2*\hv+2*\h)$) |- +(\dd,-2*\hv);
% %--------------------------------------------------------------------------
% %--------------------------------------------------------------------------
% \coordinate(dl2) at ($(dl1)+(0,-12*\hv)$) {};
% \coordinate(fl2) at ($(dl2)+(15.5,0)$)    {};
% \node[mot] (frm) at ($(dl2)+(0.8,0)$)     {\bf FROM};
% \node[var] (tb1) at ($(frm)+(1.8,0)$)     {\tt table};
% \node[mot] (as2) at ($(tb1)+(1.6,-2*\h)$) {\bf AS};
% \node[var] (nm2) at ($(as2)+(1.2,0)$)     {\tt alias};
% \coordinate(ml2) at ($(nm2.east)+(4mm,2*\h)$)  {};
% \node[mot] (joi) at ($(ml2)+(4.5*\dd,2*\hv)$){\bf JOIN};
% \node[var] (tb2) at ($(joi)+(1.6,0)$)    {\tt table};
% \node[mot] (as3) at ($(tb2)+(1.6,-2*\h)$){\bf AS};
% \node[var] (nm3) at ($(as3)+(1.2,0)$)    {\tt alias};
% \node[mot] (on_) at ($(nm3)+(1.6,2*\h)$) {\bf ON};
% \node[var] (equ) at ($(on_)+(1,0)$)    {=};
% \node[mot] (vgl2) at ($(fl2)+(-8cm,6*\hv)$)   {\bf ,};
% %--------------------------------------------------------------------------
% \draw[fle] (fl1) -| +(\dd,-\dd) |- ($(dl2)+(8cm,8*\hv)$);
% \draw[fle] ($(dl2)+(8cm,8*\hv)$) -- ($(dl2)+(0,8*\hv)$) -| +(-\dd,-2*\hv) |- (dl2);
% \draw[drt] (dl2) -- (frm.west);
% \draw[fle] (frm.east) -- (tb1.west);
% \draw[fle] (tb1.east) -- (ml2);
% \draw[gau] (as2.west) -| +(-\dd,\h)  |- +(-2*\dd,2*\h);
% \draw[fle] (as2.east) -- (nm2.west);
% \draw[drt] (nm2.east) -| +(\dd,\h)   |- (ml2);
% \draw[fle] (ml2) -| +(\dd,\dd) |- (joi.west);
% \draw[fle] (joi.east) -- (tb2.west);
% \draw[fle] (tb2.east) -- (on_.west);
% \draw[fle] (on_.east) -- (equ.west);
% \draw[drt] (equ.east) -| ($(fl2)+(-\dd,\hv)$) |- (fl2);
% \draw[gau] (as3.west) -| +(-\dd,\h)  |- +(-2*\dd,2*\h);
% \draw[fle] (as3.east) -- (nm3.west);
% \draw[drt] (nm3.east) -| +(\dd,\h)   |- +(2*\dd,2*\h);
% \draw[fle] (equ.east) -| ($(fl2)+(-\dd,2*\hv+\dd)$) |- ($(fl2)+(-2*\dd,4*\hv)$) -- ($(as3.east)+(0,2*\h+2*\hv)$);
% \draw[drt] ($(as3.east)+(0,2*\h+2*\hv)$) -- ($(ml2)+(2*\dd,4*\hv)$) -| ($(ml2)+(\dd,2*\hv)$) |- (joi.west);
% \draw[fle] (ml2) -| +(\dd,-\dd) |- +(2*\dd,-2*\hv) -- ($(fl2)+(-5cm,-2*\hv)$);
% \draw[drt] ($(fl2)+(-5cm,-2*\hv)$)-- ($(fl2)+(-2*\dd,-2*\hv)$) -| ($(fl2)+(-\dd,-\hv)$) |- (fl2);
% \draw[fle] (fl2) -| +(\dd,\dd) |- (vgl2.east);
% \draw[drt] (vgl2.west) -| ($(dl2)+(1.7cm,4*\hv)$) |- (tb1.west);
% %--------------------------------------------------------------------------
% %--------------------------------------------------------------------------
% \coordinate(dl3) at ($(dl2)+(0,-10*\hv)$) {};
% \coordinate(fl3) at ($(dl3)+(7.5,0)$)    {};
% \node[mot] (whr) at ($(dl3)+(3,0)$)     {\bf WHERE};
% \node[var] (cn1) at ($(whr)+(3,0)$)     {\tt condition};
% \node[mot] (and) at ($(cn1)+(0,4*\hv)$) {\bf AND};
% \node[mot] (or_) at ($(cn1)+(0,2*\hv)$) {\bf OR};
% %--------------------------------------------------------------------------
% \draw[fle] (fl2) -| +(\dd,-\dd) |- ($(dl3)+(8cm,6*\hv)$);
% \draw[drt] ($(dl3)+(8cm,6*\hv)$) -- ($(dl3)+(0,6*\hv)$) -| +(-\dd,-2*\hv) |- (dl3);
% \draw[fle] (dl3) -- (whr.west);
% \draw[fle] (whr.east) -- (cn1.west);
% \draw[drt] (cn1.east) -- (fl3);
% \draw[fle] (cn1.east) -| +(2*\dd,\hv) |- (or_.east);
% \draw[fle] (cn1.east) -| +(2*\dd,\hv) |- (and.east);
% \draw[drt] (cn1.west) -| +(-2*\dd,\hv) |- (or_.west);
% \draw[gau] (cn1.west) -| +(-2*\dd,\hv) |- (and.west);
% %--------------------------------------------------------------------------
% %--------------------------------------------------------------------------
% \coordinate(dl4) at ($(dl3)+(0,-9*\hv)$) {};
% \coordinate(fl4) at ($(dl4)+(15.5,0)$)    {};
% \node[mot] (grp) at ($(dl4)+(3,0)$)     {\bf GROUP BY};
% \node[var] (cl2) at ($(grp)+(2.5,0)$)  {\tt colonne};
% \node[mot] (hvg) at ($(cl2)+(3.5,2*\hv)$)     {\bf HAVING};
% \node[var] (cn2) at ($(hvg)+(3,0)$)     {\tt condition};
% \node[mot] (and) at ($(cn2)+(0,4*\hv)$) {\bf AND};
% \node[mot] (or_) at ($(cn2)+(0,2*\hv)$) {\bf OR};
% %--------------------------------------------------------------------------
% \draw[fle] (fl3) -| +(\dd,-\dd) |- ($(dl4)+(4cm,6*\hv)$);
% \draw[drt] ($(dl4)+(4cm,6*\hv)$) -- ($(dl4)+(0,6*\hv)$) -| +(-\dd,-2*\hv) |- (dl4);
% \draw[fle] ($(dl3)+(-\dd,\hv)$) -- +(0,-3*\hv);
% \draw[drt] ($(dl3)+(-\dd,\hv)$) -- +(0,-7*\hv);
% \draw[fle] (dl4) -- (grp.west);
% \draw[fle] (grp.east) -- (cl2.west);
% \draw[fle] (cl2.east) -| +(2*\dd,-\hv) |- +(5cm,-2*\hv);
% \draw[drt] (fl4) -| +(-\dd,-\hv) |- +(-4.5cm,-2*\hv);
% \draw[fle] (cl2.east) -| +(2*\dd,\hv) |- (hvg.west);
% \draw[fle] (hvg.east) -- (cn2.west);
% %\draw[drt] (cn2.east) -- (fl3);
% \draw[fle] (cn2.east) -| +(2*\dd,\hv) |- (or_.east);
% \draw[fle] (cn2.east) -| +(2*\dd,\hv) |- (and.east);
% \draw[drt] (cn2.west) -| +(-2*\dd,\hv) |- (or_.west);
% \draw[gau] (cn2.west) -| +(-2*\dd,\hv) |- (and.west);
% \draw[gau] (fl4) -| +(-\dd,\hv) |- (cn2.east);
% %--------------------------------------------------------------------------
% %--------------------------------------------------------------------------
% \coordinate(dl5) at ($(dl4)+(0,-13*\hv)$) {};
% \coordinate(fl5) at ($(dl5)+(15.5,0)$)    {};
% \node[mot] (ord) at ($(dl5)+(1.6,3*\hv)$)   {\bf ORDER BY};
% \node[var] (cl3) at ($(ord)+(2.2,0)$)       {\tt colonne};
% \node[mot] (des) at ($(cl3)+(2,-2*\h)$)   {\bf DESC};
% \node[mot] (lim) at ($(cl3)+(4.4,2*\hv)$)   {\bf LIMIT};
% \node[var] (nb1) at ($(lim)+(1.7,0)$)       {\tt nombre};
% \node[mot] (off) at ($(nb1)+(2.2,2*\hv)$)   {\bf OFFSET};
% \node[var] (nb2) at ($(off)+(1.8,0)$)       {\tt nombre};
% %--------------------------------------------------------------------------
% \draw[fle] (fl4) -| +(\dd,-\dd) |- ($(dl5)+(8cm,9*\hv)$);
% \draw[drt] ($(dl5)+(8cm,9*\hv)$) -- ($(dl5)+(0,9*\hv)$) -| +(-\dd,-2*\hv) |- (dl5);
% \draw[fle] ($(dl4)+(-\dd,\hv)$) -- +(0,-3*\hv);
% \draw[drt] ($(dl4)+(-\dd,\hv)$) -- +(0,-7*\hv);
% \draw[fle] (dl5) -- +(8cm,0);
% \draw[fle] (dl5) -- (fl5);
% \draw[gau] (ord.west) -| +(-\dd,-\hv) |- +(-2*\dd,-3*\hv);
% \draw[fle] (ord.east) -- (cl3.west);
% \draw[fle] (cl3.east) -- +(7cm,0);
% \draw[fle] (cl3.east)-| ($(fl5)+(-2*\dd,\hv)$) |- (fl5);
% \draw[gau] (des.west) -| +(-\dd,\h) |- (cl3.east);
% \draw[drt] (des.east) -| +(\dd,\h) |- +(2*\dd,2*\h);
% \draw[gau] (lim.west) -| +(-\dd,-\hv) |- (cl3.east);
% \draw[fle] (lim.east) -- (nb1.west);
% \draw[drt] (nb1.east) -| ($(fl5)+(-2*\dd,3*\hv)$) |- (fl5);;
% \draw[gau] (off.west) -| +(-\dd,-\hv) |- (nb1.east);
% \draw[fle] (off.east) -- (nb2.west);
% \draw[drt] (nb2.east) -| ($(fl5)+(-2*\dd,5*\hv)$) |- (fl5);;
% %--------------------------------------------------------------------------
% %--------------------------------------------------------------------------
% \coordinate(dl6) at ($(dl5)+(0,-4*\hv)$) {};
% \node[mot] (uni) at ($(dl6)+(3,0)$)   {\bf UNION};
% \node[mot] (int) at ($(uni)+(0,2*\hv)$)   {\bf INTERSECTION};
% \node[mot] (dif) at ($(uni)+(0,-2*\hv)$)   {\bf EXCEPT};
% %--------------------------------------------------------------------------
% \draw[fle] ($(dl5)+(-\dd,\hv)$) |- (uni.west);
% \draw[fle] ($(uni.west)+(-15mm,0)$) -| +(\dd,\dd) |- (int.west);
% \draw[fle] ($(uni.west)+(-15mm,0)$) -| +(\dd,-\dd) |- (dif.west);
% \draw[gau] ($(uni.east)+(15mm,0)$) -| +(-\dd,\dd) |- (int.east);
% \draw[drt] ($(uni.east)+(15mm,0)$) -| +(-\dd,-\dd) |- (dif.east);
% \draw[fle] (uni.east) -| +(15mm+\dd,-\dd) |- ($(dl5)+(0,-8*\hv)$) -| ($(dl4)+(-3*\dd,0)$);
% \draw[drt] ($(dl4)+(-3*\dd,0)$) |- (dl1);
% \endtikzpicture
%--------------------------------------------------------------------------
%--------------------------------------------------------------------------
%--------------------------------------------------------------------------
\section{Complément : championnat de France}
%--------------------------------------------------------------------------
%--------------------------------------------------------------------------
{\it Nous allons travailler avec une base de données qui recense les renseignements sur les matchs de division 1 d'une année.
La base ne contient qu'une table, nommée \type{statFoot}.

\medskip

En voici quelques lignes, la saison est 2018-2019
\begin{center}
\begin{tabular}{|c|c|c|c|c|c|c|c|c|}
\hline
jour       &equDom        &equExt    &butsDom &butsExt \\%&buts1D &buts1E\\
\hline
10/08/2018 & Marseille    & Toulouse & 4 & 0\\
11/08/2018 & Angers       & Nimes    & 3 & 4\\
11/08/2018 & Lille        & Rennes   & 3 & 1\\
11/08/2018 & Montpellier  & Dijon    & 1 & 2\\
11/08/2018 & Nantes       & Monaco   & 1 & 3\\
\end{tabular}
\end{center}
Il y a 20 équipes ; chaque équipe reçoit toutes les autres une fois et une seule.

\begin{itemize}
\item \type{jour} est la date du match, sous forme d'une chaîne de caractères.
\item Les autres attributs vont par paire : le suffixe \type{Dom} indique que l'attribut concerne l'équipe qui joue à domicile, le suffixe \type{Ext} indique que l'attribut concerne l'équipe qui joue à l'extérieur.
\item \type{equDom} ou \type{equExt} indique le nom de l'équipe, sous forme d'une chaîne de caractères.
\item \type{butsDom} ou \type{butsExt} indique le nombre de buts marqués pendant le match.
\end{itemize}

Un match nul vaut 1 point, un match gagné vaut 3 points.
}
%--------------------------------------------------------------------------
%--------------------------------------------------------------------------
\begin{Exercise}
Écrire une requête qui permet de connaître le nombre de matchs gagnés par chaque équipe, le nombre de matchs nuls et le total des points de l'année. 

N.B. Bien que la base ne contienne qu'une table, la requête peut faire intervenir des jointures.
\end{Exercise}
%--------------------------------------------------------------------------
\begin{Answer}
\begin{lstlisting}[language=SQL]
select d.club, (d.victoiresDom + e.victoiresExt) as victoires,
       n.nuls, 
       (38 - n. nuls - d.victoiresDom - e.victoiresExt) as Défaites, 
       (3*(d.victoiresDom + e.victoiresExt) + n.nuls) as points
from (select equDom as club, count() as victoiresDom
      from stat 
      where butsDom > butsExt group by equDom) as d
	 join
	 (select equExt as club, count() as victoiresExt
      from stat 
      where butsDom < butsExt group by equExt) as e
	 join
	 (select equDom as club, count() as nuls
      from stat 
      where butsDom = butsExt group by equDom) as n
	 on e.club = d.club and e.club = n.club
order by points desc
\end{lstlisting}
\end{Answer}
%--------------------------------------------------------------------------
\newpage
%--------------------------------------------------------------------------
%--------------------------------------------------------------------------
%--------------------------------------------------------------------------

