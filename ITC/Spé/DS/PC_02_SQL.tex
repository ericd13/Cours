%-------------------------------------------------------------------------------
%-------------------------------------------------------------------------------
%-------------------------------------------------------------------------------
\chapter{Bases de données}
%-------------------------------------------------------------------------------
%-------------------------------------------------------------------------------
%-------------------------------------------------------------------------------
\section{Présentation} 
%-------------------------------------------------------------------------------
%-------------------------------------------------------------------------------
%-------------------------------------------------------------------------------
Les questions de ce devoir porteront sur la base de données étudiée en T.P.

On utilisera les tables {\sc Pays}, {\sc Villes}, {\sc Montagne} et {\sc MontagnePays}. 

La table {\sc Pays} contient les attributs
\begin{itemize}
    \item {\bf nom} est le nom du pays,
    \item {\bf capitale} est le nom de la capitale du pays,    
    \item {\bf superficie} est la surface du pays,
    \item {\bf population} est le nombre d'habitants du pays,
    \item {\bf code} est le nom abrégé du pays qui sera utilisé dans les autres tables.
\end{itemize}

La table {\sc Ville} contient les attributs
\begin{itemize}
    \item {\bf nom} est le nom de la ville,
    \item {\bf population} est le nombre d'habitants de la ville,
    \item {\bf pays} est le {\bf code} du pays dans lequel est située la ville.
\end{itemize}

La table {\sc Montagne} contient les attributs
\begin{itemize}
    \item {\bf nom} est le nom de la montagne,
    \item {\bf chaine} est la chaîne de montagnes contenant la montagne (Alpes, Himalaya, \dots),
    \item {\bf altitude} est l'altitude du sommet.
\end{itemize}

La table {\sc MontPays} relie les montagnes et les pays
\begin{itemize}
    \item {\bf montagne} est le nom de la montagne,
    \item {\bf pays} est le {\bf code} du pays dans lequel est située la montagne.
\end{itemize}
Une montagne peut être dans plusieurs pays et, bien sur un pays peut contenir plusieurs montagnes.

Les liaisons naturelles, qui peuvent servir dans les jointures, sont indiquées dans le diagramme ci-dessous.

\medskip
%-------------------------------------------------------------------------------
\begin{center}
\tikzstyle{table}=[draw,shape=rectangle,text width=20mm,align=center,minimum height=6mm]
\tikzpicture
\node[table] at (0, 0.0) {\sc Ville};
\node[table] at (0,-0.6) (capv) {nom};
\node[table] at (0,-1.2) {population};
\node[table] at (0,-1.8) (pv){pays};
\node[table] at (4, 0.0) {\sc Pays};
\node[table] at (4,-0.6) {nom};
\node[table] at (4,-1.2) (cap) {capitale};
\node[table] at (4,-1.8) {superficie};
\node[table] at (4,-2.4) {population};
\node[table] at (4,-3.0) (code) {code};
\node[table] at (8, 0.0) {\sc MontPays};
\node[table] at (8,-0.6) (mpm) {montagne};
\node[table] at (8,-1.2) (mpp) {pays};
\node[table] at (12, 0.0) {\sc Montagne};
\node[table] at (12,-0.6) (mon) {nom};
\node[table] at (12,-1.2) {chaine};
\node[table] at (12,-1.8) {altitude};
\draw[thick,dotted, <->] (capv) -- +(2, 0) |-  (cap);
\draw[thick, <->] (pv) -- +(2, 0) |-  (code);
\draw[thick, <->] (code) -- +(2 , 0) |-  (mpp);
\draw[thick, <->] (mpm) -- +(2 , 0) |-  (mon);
\endtikzpicture
\end{center}
%--------------------------------------------------------------------------
\newpage
%-------------------------------------------------------------------------------
%-------------------------------------------------------------------------------
%-------------------------------------------------------------------------------
\section{Requêtes} 
%-------------------------------------------------------------------------------
%-------------------------------------------------------------------------------
%-------------------------------------------------------------------------------
\subsection{Requêtes simples} 
%-------------------------------------------------------------------------------
%-------------------------------------------------------------------------------
\begin{Exercise}
\it Écrire une requête qui renvoie le nom et l'altitude des montagnes de plus de 800 mètres d'altitude
\end{Exercise}
%-------------------------------------------------------------------------------
\begin{Answer}
\begin{lstlisting}[language=SQL]
SELECT nom, altitude
FROM Montagne
WHERE altitude > 8000
\end{lstlisting}
\end{Answer}
%-------------------------------------------------------------------------------
%-------------------------------------------------------------------------------
\begin{Exercise}
\it Écrire une requête qui renvoie le nom des villes de plus de 5 millions (5e6) d'habitants ainsi que leur population.
\end{Exercise}
%-------------------------------------------------------------------------------
\begin{Answer}
\begin{lstlisting}[language=SQL]
SELECT nom, population
FROM Ville
WHERE population > 5e9
\end{lstlisting}
\end{Answer}
%-------------------------------------------------------------------------------
%-------------------------------------------------------------------------------
\begin{Exercise}
\it Quels sont les pays dont la densité (population/superficie) est supérieure à 500 ?
\end{Exercise}
%-------------------------------------------------------------------------------
\begin{Answer}
\begin{lstlisting}[language=SQL]
SELECT nom, population
FROM Ville
WHERE population > 5e9
\end{lstlisting}
\end{Answer}
%-------------------------------------------------------------------------------
%-------------------------------------------------------------------------------
\subsection{Agrégations} 
%-------------------------------------------------------------------------------
%-------------------------------------------------------------------------------
\begin{Exercise}
\it Quelle est la moyenne (\type{AVG}) de la population des "gros" pays, ceux ayant plus de 10 millions d'habitants ?
\end{Exercise}
%-------------------------------------------------------------------------------
\begin{Answer}
\begin{lstlisting}[language=SQL]
SELECT AVG(population)
FROM Pays
WHERE population > 1e7
\end{lstlisting}
\end{Answer}
%-------------------------------------------------------------------------------
%-------------------------------------------------------------------------------
\begin{Exercise}
\it Écrire une requête qui renvoie le nom des chaînes de montagne ainsi que le nombre de sommets leur appartenant.
\end{Exercise}
%-------------------------------------------------------------------------------
\begin{Answer}
\begin{lstlisting}[language=SQL]
SELECT chaine, count()
FROM Montagne
GROUP BY chaines
\end{lstlisting}
\end{Answer}
%-------------------------------------------------------------------------------
%-------------------------------------------------------------------------------
\begin{Exercise}
\it Quels sont les code des pays contenant au moins 5 villes de plus de 1 million d'habitants ?
\end{Exercise}
%-------------------------------------------------------------------------------
\begin{Answer}
\begin{lstlisting}[language=SQL]
SELECT pays, count() AS nb_villes
FROM Ville
Where population > 1e6
GROUP BY pays
HAVING nb_villes >= 5
\end{lstlisting}
\end{Answer}
%-------------------------------------------------------------------------------
%-------------------------------------------------------------------------------
\subsection{Jointures} 
%-------------------------------------------------------------------------------
%-------------------------------------------------------------------------------
\begin{Exercise}
\it Quels sont les pays dont la capitale a plus de 5 millions d'habitants ?
\end{Exercise}
%-------------------------------------------------------------------------------
\begin{Answer}
\begin{lstlisting}[language=SQL]
SELECT p.nom, v.Population
FROM pays AS p JOIN ville AS v on p.capitale = v.Nom
where v.population > 5e6
\end{lstlisting}
\end{Answer}
%-------------------------------------------------------------------------------
%-------------------------------------------------------------------------------
\begin{Exercise}
\it Quelles sont les montagnes en France ? On donnera aussi leur altitude.
\end{Exercise}
%-------------------------------------------------------------------------------
\begin{Answer}
\begin{lstlisting}[language=SQL]
SELECT m.nom, m.Altitude
FROM Pays AS p JOIN MontaPays AS mp ON p.code = mp.pays
               JOIN Montagnes AS m ON m.nom = mp.montagne
WHERE p.nom = "France"
\end{lstlisting}
\end{Answer}
%-------------------------------------------------------------------------------
%-------------------------------------------------------------------------------
\begin{Exercise}
\it Quels sont les pays qui contiennent des montagnes de plus de 8000 mètres ?
\end{Exercise}
%-------------------------------------------------------------------------------
\begin{Answer}
\begin{lstlisting}[language=SQL]
SELECT p.nom, m.nom, m.Altitude
FROM Pays AS p JOIN MontPays AS mp ON p.code = mp.pays
               JOIN Montagne AS m ON m.nom = mp.montagne
WHERE m.altitude > 8000
\end{lstlisting}
\end{Answer}
%-------------------------------------------------------------------------------
%-------------------------------------------------------------------------------
\begin{Exercise}
\it Quels sont les pays qui contiennent une montagne de la chaîne des Alpes (\type{"Alps"}) ?
\end{Exercise}
%-------------------------------------------------------------------------------
\begin{Answer}
\begin{lstlisting}[language=SQL]
SELECT DISTINCT p.nom
FROM Pays AS p JOIN MontPays AS mp ON p.code = mp.pays
               JOIN Montagne AS m ON m.nom = mp.montagne
WHERE m.chaine = "Alps"
\end{lstlisting}
\end{Answer}
%-------------------------------------------------------------------------------
%-------------------------------------------------------------------------------
\begin{Exercise}
\it Quels sont les pays qui contiennent au moins 5 montagnes de plus de 6000 mètres ?
\end{Exercise}
%-------------------------------------------------------------------------------
\begin{Answer}
\begin{lstlisting}[language=SQL]
SELECT p.nom, count() AS nombre
FROM Pays AS p JOIN MontPays AS mp ON p.code = mp.pays
               JOIN Montagne AS m ON m.nom = mp.montagne
WHERE m.altitude > 6000
GROUP BY p.nom
HAVING nombre >= 5
\end{lstlisting}
\end{Answer}
%-------------------------------------------------------------------------------
%-------------------------------------------------------------------------------
\subsection{Plus} 
%-------------------------------------------------------------------------------
%-------------------------------------------------------------------------------
\begin{Exercise}
\it Quels sont les pays dont la population est supérieure à 10 fois la moyenne de la population des gros pays (question 4) ?
\end{Exercise}
%-------------------------------------------------------------------------------
\begin{Answer}
\begin{lstlisting}[language=SQL]
SELECT nom
FROM pays
WHERE population >= 10*(SELECT AVG(population)
                        FROM Pays
                        WHERE population > 1e7)
\end{lstlisting}
\end{Answer}
%-------------------------------------------------------------------------------
%-------------------------------------------------------------------------------
\begin{Exercise}
\it Quels sont les pays pour lesquels la capitale n'est pas la ville la plus peuplée ?
\end{Exercise}
%-------------------------------------------------------------------------------
\begin{Answer}
On commence par créer la table des pays avec la population maximale d'une ville
\begin{lstlisting}[language=SQL]
SELECT p.nom AS pays, MAX(v.population) AS popmax
FROM Pays AS p JOIN Ville AS v ON v.pays = p.code
GROUP BY p.nom
\end{lstlisting}

Cette table va être jointe à \type{Pays} elle même jointe à \type{Villes}
\begin{lstlisting}[language=SQL]
SELECT p.nom, p.capitale, v.population, m.popmax
FROM Pays AS p 
     JOIN Ville AS v ON v.nom = p.capitale AND v.pays = p.code
     JOIN (SELECT p.nom AS pays, MAX(v.population) AS popmax
           FROM Pays AS p JOIN Ville AS v ON v.pays = p.code
           GROUP BY p.nom) AS m ON m.pays = p.nom
WHERE v.population <> m.popmax
\end{lstlisting}
On doit ajouter la condition \type{AND v.pays = p.code} à la ligne 2 car il y a plusieurs villes qui ont le même nom.
\newpage
Si on veut en plus, le nom de la ville la plus peuplée, c'est plus difficile.
\begin{lstlisting}[language=SQL,numbers=left]
SELECT p.nom, p.capitale, v.population AS population_c, 
       vm.ville_m, vm.population_m
FROM (SELECT m.pays AS pays, v.nom AS ville_m, 
             m.popmax AS population_m
        FROM Ville AS v 
             JOIN (SELECT p.nom AS pays, p.code AS code, 
                   MAX(v.population) AS popmax
                   FROM Pays AS p 
                        JOIN Ville AS v ON v.pays = p.code
                   GROUP BY p.nom) AS m ON m.code = v.Pays
        WHERE v.population = m.popmax) AS vm
     JOIN Pays AS p ON vm.pays = p.nom
     JOIN Ville AS v ON v.nom = p.capitale and v.pays = p.code
WHERE p.capitale <> vm.ville_m AND population_c IS NOT NULL
\end{lstlisting}
On doit ajouter la condition \type{AND population\_c is not null} à la ligne 14 car il y a des capitales dont la population n'est pas connue
\end{Answer}
%-------------------------------------------------------------------------------
%-------------------------------------------------------------------------------
