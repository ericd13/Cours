%-------------------------------------------------------------------------------
%-------------------------------------------------------------------------------
%-------------------------------------------------------------------------------
\chapter{Listes : 2}
%-------------------------------------------------------------------------------
%TODO filtrage de listes
%-------------------------------------------------------------------------------
\thispagestyle{empty}
%-------------------------------------------------------------------------------
\begin{abstract}
Nous avons défini le type \type{list} de python et les fonctions de base : on les retrouve dans la plupart des langages de programmation, avec un type très souvent appelé {\bf tableau} ({\it array}). Dans ces langages apparaît souvent un autre type d'assemblage, appelé liste, qui se construit et s'utilise pas-à-pas. Dans le langage python ces deux types sont combinés et nous allons étudier les actions supplémentaires sur les listes.
\end{abstract}
%-------------------------------------------------------------------------------
%-------------------------------------------------------------------------------
\section{Création de liste par augmentations}
%-------------------------------------------------------------------------------
%-------------------------------------------------------------------------------
\subsection{Définition}
%-------------------------------------------------------------------------------
%-------------------------------------------------------------------------------
Jusqu'à présent nous utilisions les listes avec une longueur fixée : on pouvait changer les valeurs de chacune des positions mais on ne pouvait pas ajouter un élément supplémentaire.

On peut créer une {\bf nouvelle} liste avec un élément de plus :
%-------------------------------------------------------------------------------
\begin{lstlisting}
>>> l = [4, 3, 7, 6, 0]
>>> l1 = l + [5]
>>> l1
[4, 3, 7, 6, 0, 5]
>>> l
[4, 3, 7, 6, 0]
\end{lstlisting}
%-------------------------------------------------------------------------------
Cette opération ne modifie pas la liste initiale mais crée une nouvelle liste : cela nécessite de copier tous les éléments et a donc un coût que l'on souhaite éviter.

Python a prévu cette possibilité sous la forme d'une méthode.

%-------------------------------------------------------------------------------
\begin{defin}[Méthode]
Une méthode est une fonction d'un type spécial qui est invoquée en ajoutant son nom {\bf après} l'objet auquel elle s'applique avec un point de séparation. Une méthode peut avoir des paramètres, ils seront placés classiquement entre les parenthèses qui suivent le nom de la méthode.
\end{defin}
%-------------------------------------------------------------------------------
\begin{defin}[append]
L'instruction \Type{liste.append(x)} attache l'élément \type{x} au bout de la liste \type{liste}. C'est une instruction élémentaire qui procède sans affectation, elle ne fait que modifier sans créer.
\end{defin}
%-------------------------------------------------------------------------------
\begin{lstlisting}
>>> l = [4, 3, 7, 6, 0]
>>> l.append(5)
>>> l
[4, 3, 7, 6, 0, 5]
\end{lstlisting}
%-------------------------------------------------------------------------------
Cette opération a un coût constant\footnote{Cela n'est vrai qu'en moyenne, on parle de coût amorti.}, il ne dépend pas de la longueur de la liste initiale.
%-------------------------------------------------------------------------------
%-------------------------------------------------------------------------------
\subsection{Usage}
%-------------------------------------------------------------------------------
%-------------------------------------------------------------------------------
On obtient ainsi une autre méthode pour construire une liste.

En voici quelques exemples
%-------------------------------------------------------------------------------
\subsubsection{Application d'une fonction aux éléments d'une liste}
%-------------------------------------------------------------------------------
\begin{multicols}{2}
\setlength{\columnseprule}{1pt}

\begin{lstlisting}[frame=no]
def carre(liste):
    n = len(liste)
    l2 = [0]*n
    for x in liste:
        l2[i] = x**2
    return l2
\end{lstlisting}

\begin{lstlisting}[frame=no]
def carre(liste):
    l2 = []
    for x in range(n):
        l2.append(x**2)
    return l2
\end{lstlisting}
\end{multicols}

On voit que, même si on utilise la lecture directe des éléments de la liste (\type{for x in liste}), la méthode classique nécessite de connaître la longueur de la liste. Cependant la méthode la plus directe est d'utiliser la construction Python :
\begin{lstlisting}
def carre(liste):
    return [x**2 for x in liste]
\end{lstlisting}
%-------------------------------------------------------------------------------
\subsubsection{Calcul des termes d'une suite}
%-------------------------------------------------------------------------------
\begin{multicols}{2}
\setlength{\columnseprule}{1pt}

\begin{lstlisting}[frame=no]
def fibonacci(n):
    F = [0]*(n+1)
    F[0] = 0
    F[1] = 1
    for i in range(2, n+1):
        F[i] = F[i-1] + F[i-2]
    return F
\end{lstlisting}

\columnbreak 

\begin{lstlisting}[frame=no]
def fibonacci(n):
    F = [0, 1]
    for i in range(2, n+1):
        F.append(F[i-1] + F[i-2])
    return F
\end{lstlisting}
\end{multicols}

Dans le cas des adjonction on peut remplacer le calcul par \type{ F.append(F[-1] + F[-2])}.
%-------------------------------------------------------------------------------
\subsubsection{Filtrage d'une liste}
%-------------------------------------------------------------------------------
On veut extraire d'une liste ses termes positifs.
%-------------------------------------------------------------------------------
\begin{lstlisting}
def positifs(liste):
    pos = []
    for x in liste:
        if x >= 0:
            pos.append(x)
    return pos
\end{lstlisting}
%-------------------------------------------------------------------------------
Ici l'écriture classique n'est pas raisonnable, il faudrait commencer par calculer le nombres de termes positifs puis les écrire dans une liste de taille adaptée. 
%-------------------------------------------------------------------------------
%-------------------------------------------------------------------------------
\begin{Exercise}\it 
Écrire cette fonction.
\end{Exercise}
%-------------------------------------------------------------------------------
\begin{Answer}
\begin{lstlisting}
def positifs(liste):
    nb_pos = 0
    for x in liste:
        if x >= 0:
            nb_pos = nb_pos + 1
    pos = [0]*nb_pos
    k = 0
    for x in liste:
        if x >= 0:
            pos[k] = x
            k = k + 1
    return pos
\end{lstlisting}
\end{Answer}
%-------------------------------------------------------------------------------
%-------------------------------------------------------------------------------
Ici encore, Python fournit une construction très efficace.
%-------------------------------------------------------------------------------
\begin{lstlisting}
def positifs(liste):
    return [x for x in liste if x >= 0]
\end{lstlisting}
%-------------------------------------------------------------------------------
%-------------------------------------------------------------------------------
\section{Autres méthodes}
%-------------------------------------------------------------------------------
%-------------------------------------------------------------------------------

L'opération inverse de \type{append} est notée \type{pop} ; l'instruction \type{liste.pop()} a deux effets
%-------------------------------------------------------------------------------
\begin{enumerate}
\item elle modifie la liste en lui enlevant son dernier élément,
\item elle renvoie ce dernier élément. Il sera souvent utile de conserver ce dernier élément dans une variable : 

\type{a = liste.pop()}.
\end{enumerate}
%-------------------------------------------------------------------------------
Il est indispensable d'écrire les parenthèses.

\medskip

Il existe d'autres méthodes qui modifient une liste. Ce ne sont plus des opérations élémentaires, leur temps d'exécution dépend de la taille.
%-------------------------------------------------------------------------------
\begin{enumerate}
\item \type{liste.pop(k)} qui enlève l'élément d'indice $k$ de la liste et le renvoie. 
\item \type{liste.remove(x)} qui enlève la première apparition de $x$ dans la liste. Si $x$ n'est pas présent dans la liste, la méthode renvoie une erreur.
\item \type{liste.insert(i,x)} qui insère $x$ dans la liste à la position $i$ en décalant les termes suivants. 
\item \type{liste.reverse()} qui retourne la liste. 
\item \type{liste.sort()} qui trie la liste. 
\end{enumerate}
%-------------------------------------------------------------------------------


\medskip

On peut simuler le comportement de \type{liste.pop(k)} à l'aide d'une fonction.
%-------------------------------------------------------------------------------
\begin{lstlisting}
def enlever(k, liste):
    a = liste[k]
    n = len(liste)
    for i in range(k, n-1):
        liste[i] = liste[i+1]
    liste.pop()
    return a
\end{lstlisting}
%-------------------------------------------------------------------------------
On voit qu'on effectue $n-1=k$ affectations, le temps d'exécution dépend de la longueur.
%-------------------------------------------------------------------------------
%-------------------------------------------------------------------------------
\begin{Exercise}\it 
Simuler de même les méthodes \type{remove} et \type{insert}.
\end{Exercise}
%-------------------------------------------------------------------------------
\begin{Answer}

\medskip

\begin{minipage}{0.53\textwidth}
\begin{lstlisting}
def oter(x, liste):
    n = len(liste):
    k = 0
    while k < n 
      and liste[k] != x:
        k = k + 1
    if k < n:
        for i in range(k, n-1):
          liste[i] = liste[i+1]
        liste.pop()
\end{lstlisting}
\end{minipage}
\begin{minipage}{0.47\textwidth}
\begin{lstlisting}
def inserer(x, k, liste):
    n = len(liste):
    liste.append(x)
    for i in range(n, k, -1):
        liste[i] = liste[i-1]
    liste[k] = x
\end{lstlisting}
\end{minipage}
\end{Answer}
%-------------------------------------------------------------------------------
%-------------------------------------------------------------------------------
\section{Décomposition en base 2}
%-------------------------------------------------------------------------------
%-------------------------------------------------------------------------------
La représentation interne des entiers dans un ordinateurs se fait en base 2.

Elle s'appuie sur la possibilité d'écrire tout entier positif $n < 2^{p+1}$ de manière unique sous la forme
\[n = \sum_{k=0}^p \epsilon_k2^k\hbox{ avec } \epsilon_k\in\{0, 1\}\]
Pour calculer les $\epsilon_k$, on utilise les propriétés suivantes
\begin{enumerate}
    \item Comme $2^k$ est pair pour $k\ge 0$, $\epsilon_0$ vaut 1 si et seulement si $n$ est impair, c'est le {\bf bit de parité}. $\epsilon_0$ est donc la valeur de \type{n\%2}.
    \item Pour $0\le q\le p$, la décomposition de la division entière de $n$ par $2^q$ est $\displaystyle \sum_{k=q}^p \epsilon_k2^{k-q}$. 
    
    $\epsilon_q$ est donc la valeur de \type{(n//2**q)\%2}.
    \item Pour $q>p$ la division entière de $n$ par $2^q$ donne 0, il n'y a plus de composante à calculer. On peut donc calculer la liste des $\epsilon_k$ utiles sans connaître $p$ à l'avance en s'arrêtant dès que le quotient \type{n//2**q} vaut 0.
\end{enumerate}

\medskip

On va donc calculer la liste des coefficients \type[e0, e1, ..., ep] par adjonctions successives. On remarque que l'on obtient le coefficient de $2^0$ (de poids faible) en premier ; si on a besoin des coefficients dans l'ordre inverse, on retournera la liste obtenue.

\medskip

Voici une première écriture possible.
%-------------------------------------------------------------------------------
\begin{lstlisting}
def base2(n):
    b2 = []
    q = 0
    while n//2**q > 0:
        e = (n//2**q)%2
        b2.append(e)
        q = q + 1
    return b2
\end{lstlisting}
%-------------------------------------------------------------------------------

\medskip

On remarque que l'on fait des calculs inutiles :

\begin{enumerate}
    \item \type{n//2**q} est calculé deux fois,
    \item plutôt que le calculer, on peut passer de \type{n//2**q} à \type{n//2**(q+1)} en divisant par 2.
\end{enumerate}

\medskip
%-------------------------------------------------------------------------------
\begin{lstlisting}
def base2(n):
    b2 = []
    while n > 0:
        e = n%2
        b2.append(e)
        n = n//2
    return b2
\end{lstlisting}
%-------------------------------------------------------------------------------
%-------------------------------------------------------------------------------
%-------------------------------------------------------------------------------
\section{Solutions} 
%-------------------------------------------------------------------------------
%-------------------------------------------------------------------------------
%-------------------------------------------------------------------------------
\shipoutAnswer
