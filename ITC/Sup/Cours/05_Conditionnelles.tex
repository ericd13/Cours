%-------------------------------------------------------------------------------
%-------------------------------------------------------------------------------
%-------------------------------------------------------------------------------
\chapter{Les structures conditionnelles}
%-------------------------------------------------------------------------------
%-------------------------------------------------------------------------------
\thispagestyle{empty}
%-------------------------------------------------------------------------------
%-------------------------------------------------------------------------------
\begin{abstract}
Pour l'instant nous n'avons vu les programmes que sous la forme de suites d'instructions exécutées l'une après l'autre, en répétant plusieurs fois une partie d'entre elles.

On est parfois amené à n'appliquer une partie de programme que si une condition est vérifiée ou à exécuter des traitements différents selon l'appartenance de valeurs à certains ensembles.
\end{abstract}
%-------------------------------------------------------------------------------
%-------------------------------------------------------------------------------
\section{Les structures conditionnelles}
%-------------------------------------------------------------------------------
%-------------------------------------------------------------------------------
\subsection{Instruction if}
%-------------------------------------------------------------------------------
L'instruction \Type{if} permet de vérifier des conditions avant d'exécuter un bloc d'instructions. C'est la forme la plus simple des instructions conditionnelles.
%-------------------------------------------------------------------------------
\begin{defin}[condition simple]
La structure conditionnelle est composée par 
\begin{itemize}
 \item l'instruction \type{if} suivi d'une expression booléenne puis de \type{:},
 \item un bloc d'instructions indenté, ces instructions ne seront exécutées que si l'expression booléenne est évaluée avec la valeur \type{True} (vrai).
\end{itemize}
\end{defin}
%-------------------------------------------------------------------------------
Nous préciserons dans la suite la notion d'expression booléenne.

L'indentation s'ajoute à l'indentation déjà existante au moment de l'écriture de l'instruction \type{if}.

Exemple : valeur absolue (elle existe déjà sous le nom \type{abs}).
%-------------------------------------------------------------------------------
\begin{lstlisting}
def valeur_absolue(x):
    if x < 0:
        x = -x
    return x
\end{lstlisting}
%-------------------------------------------------------------------------------
\newpage
%-------------------------------------------------------------------------------
\subsection{Instruction else}
%-------------------------------------------------------------------------------
Parfois le test (expression booléenne) sert à discriminer : on effectue des instructions si la condition est vérifiée et d'autres instructions si elle ne l'est pas.
%-------------------------------------------------------------------------------
\begin{defin}[condition complète]
La structure conditionnelle complète est composée par 
\begin{itemize}
 \item l'instruction \type{if} suivi d'une expression booléenne puis de \type{:},
 \item un bloc d'instructions indenté qui sera effectué si la l'expression booléenne vaut \type{True},
 \item \type{else:} au même niveau d'indentation que \type{if},
  \item un bloc d'instructions indenté qui sera effectué si la l'expression booléenne vaut \type{False}.
\end{itemize}
\end{defin}
%-------------------------------------------------------------------------------
Un des deux blocs (et un seul) sera toujours exécuté. On pourra donc définir une nouvelle variable en donnant sa valeur selon les cas.
%-------------------------------------------------------------------------------
\begin{lstlisting}
def valeur_absolue(x):
    if x < 0:
        val_abs = -x
    else:
        val_abs = x
    return val_abs
\end{lstlisting}
%-------------------------------------------------------------------------------
Les deux branches de l'alternative peuvent contenir une instruction \type{return}.
%-------------------------------------------------------------------------------
\begin{lstlisting}
def valeur_absolue(x):
    if x < 0:
        return -x
    else:
        return x
\end{lstlisting}
%-------------------------------------------------------------------------------
En fait, comme une instruction \type{return} interrompt la fonction on peut écrire le programme précédent sous la forme suivante, sans doute moins lisible que les autres.
%-------------------------------------------------------------------------------
\begin{lstlisting}
def valeur_absolue(x):
    if x < 0:
        return -x
    return x
\end{lstlisting}
%-------------------------------------------------------------------------------
%-------------------------------------------------------------------------------
\subsection{Instruction elif}
%-------------------------------------------------------------------------------
Il arrivera régulièrement que l'on ait à distinguer plusieurs cas : il faut alors imbriquer plusieurs instructions conditionnelles
%-------------------------------------------------------------------------------
\begin{lstlisting}
def signe(x):
    """Entrée : un nombre x
       Sortie : 1, 0 ou -1 selon que x est 
                positif, nul ou négatif"""
    if x == 0:
        return 0
    else:
        if x > 0:
            return 1
        else:
            return -1
\end{lstlisting}
%-------------------------------------------------------------------------------
mais les choses peuvent devenir compliquées quand le nombre de cas augmente.

Le programme suivant détermine la mention du bac.
%-------------------------------------------------------------------------------
\begin{lstlisting}
def afficheMention(note):
    """Entrée : la note du bac
       Sortie : la mention obtenue"""
    if note < 10:
        return 'Non admis'
    else:
        if note < 12:
            return 'Admis sans mention'
        else:
            if note < 14:
                return 'Mention assez bien'
            else:
                if note < 16:
                    return 'Mention bien'
                else:
                    return 'Mention tres bien'
\end{lstlisting}
%-------------------------------------------------------------------------------
Python propose une structure raccourcie\footnote{On parle de {\bf sucre syntaxique}.} qui évite d'imbriquer les conditions. 
%-------------------------------------------------------------------------------
\begin{defin}[séparations de cas]
On peut traiter des cas qui s'excluent par 
\begin{itemize}
 \item l'instruction \type{if} suivi d'une expression booléenne puis de \type{:},
 \item un bloc d'instructions indenté qui sera effectué si la l'expression booléenne vaut \type{True}
 \item un certain nombre d'instructions \type{elif} suivies d'une expression booléenne puis du symbole \type{:}, toutes au même niveau d'indentation que \type{if}
  \item pour chacune un bloc d'instructions indenté qui sera effectué si la l'expression booléenne vaut \type{True} et les précédentes valent \type{False}
    \item \type{else:} au même niveau d'indentation que \type{if}
  \item un bloc d'instructions indenté qui sera effectué si toutes les expressions booléennes valent \type{False}
\end{itemize}
\end{defin}
%-------------------------------------------------------------------------------
On peut donc écrire l'algorithme des mentions de manière plus lisible
%-------------------------------------------------------------------------------
\begin{lstlisting}
def mention(note):
    """Entrée : la note du bac
       Sortie : la mention obtenue"""
    if note < 10:
        return 'Non admis'
    elif note < 12:
        return 'Admis sans mention'
    elif note < 14:
        return 'Mention assez bien'
    elif note < 16:
        return 'Mention bien'
    else:
        return 'Mention très bien'
\end{lstlisting}
%-------------------------------------------------------------------------------
\subsection{Complexité : 2}
%-------------------------------------------------------------------------------
Les instructions conditionnelles introduisent une incertitude lors du calcul du nombre d'instructions ; en effet ce nombre peut être différent selon le résultat du test. On est donc amené à calculer un {\bf majorant} de la complexité, plutôt qu'une expression de celle-ci. On essaiera, dans la mesure du possible, de donner un majorant qui peut être atteint pour certaines valeurs des paramètres.
%-------------------------------------------------------------------------------
\newpage
%-------------------------------------------------------------------------------
\section{Les expressions booléennes}
%-------------------------------------------------------------------------------
%-------------------------------------------------------------------------------
Une structure conditionnelle nécessite la définition d'une \textbf{condition} dont le résultat est booléen (True ou False) : une expression booléenne. Celles-ci s'obtiennent en deux temps : on crée des résultats booléens à l'aide de comparaisons et on les combine avec des opérateurs booléens.
%-------------------------------------------------------------------------------
\subsection{Les opérateurs de comparaison}
%-------------------------------------------------------------------------------
Un premier test possible est l'identité : une variable (ou un résultat) est-il égal à une autre ?

Le test d'égalité ne s'écrit pas \type{=} qui est l'opérateur d'affectation mais \Type{==}.

On l'a utilisé pour comparer à 0 dans la fonction \type{signe}.

La différence est signifiée\footnote{On peut penser le symbole comme la déconstruction du signe mathématique $\ne$ : une barre et le signe égal.} par \Type{!=}.

\medskip

Les variables qui sont comparables, principalement les nombres, peuvent être testées selon leur ordre.
\begin{center}
\begin{tabular}{c|cc}
 &inférieur& supérieur \\
 \hline
 Strictement  & \Type{<}&\Type{>} \\
ou égal & \Type{<=}&\Type{<=} \\
\end{tabular}
\end{center}
%-------------------------------------------------------------------------------
\subsection{Les opérateurs booléens}
%-------------------------------------------------------------------------------
Les résultats des comparaisons peuvent combinés de la même manière que les nombres peuvent être combinés par les opérations usuelles d'addition, soustraction, division, \dots

Les opération de base utilisées dans python sont
%-------------------------------------------------------------------------------
\begin{itemize}
\item \Type{and} (et) : \type{test1 and test2} vaut \type{True} si et seulement si \type{test1} {\bf et} \type{test2} sont évalués à \type{True},
\item \Type{or} (ou) : \type{test1 or test2} vaut \type{True} si et seulement si au moins un des deux test est évalué à \type{True}, le ou n'est pas exclusif, si \type{test1} et \type{test2} valent \type{True} alors \type{test1 or test2} vaut \type{True}
\item \Type{not} (négation) : inverse le résultat.
\end{itemize}
%-------------------------------------------------------------------------------
\[\begin{tabular}{cc|ccc}
\type{a} & \type{b} & \type{a and b} & \type{a or b} & \type{not b}\\
\hline
\type{True} & \type{True} & \type{True} & \type{True}  & \type{False} \\
\type{True} & \type{False} & \type{False} &\type{True}  & \type{True}\\
\type{False} & \type{True} & \type{False} &\type{True} \\
\type{False} & \type{False} & \type{False} & \type{False} \\
\end{tabular}\]
%-------------------------------------------------------------------------------
\subsection{Évaluation paresseuse}

Les définitions données ci-dessus sont correctes mathématiquement mais ce n'est pas ce qu'utilisent les langages de programmation.
Lors de l'évaluation de \type{a or b} on sait que le résultat est vrai dès que la proposition \type{a} est vérifiée. Python ne calculera pas le résultat de \type{b} dans ce cas. La même remarque s'applique lors de l'évaluation de \type{a and b} : si \type{a} est faux, python renvoie \type{False} sans chercher à calculer \type{b}.

Cela rend le calcul non commutatif :
\begin{lstlisting}
>>> True or (1/0 == 1)
True

>>> (1/0 == 1) or True
ZeroDivisionError: division by zero
\end{lstlisting}
Ce comportement est en fait très utile, nous l'utiliserons avec les boucles \type{while}.
%-------------------------------------------------------------------------------
%-------------------------------------------------------------------------------
%-------------------------------------------------------------------------------
