%-------------------------------------------------------------------------------
%-------------------------------------------------------------------------------
%-------------------------------------------------------------------------------
\chapter{Module Numpy}
%-------------------------------------------------------------------------------
%-------------------------------------------------------------------------------
\thispagestyle{empty}
%--------------------------------------------------------------------------
%--------------------------------------------------------------------------
%--------------------------------------------------------------------------
\section{Listes Python}
%--------------------------------------------------------------------------
%--------------------------------------------------------------------------
%--------------------------------------------------------------------------
\subsection{Avantages}
%--------------------------------------------------------------------------
%--------------------------------------------------------------------------
Les liste python sont un type de données très souple :
\begin{itemize}
\item on accède aux éléments directement aux éléments par leur indice,
\item les valeurs sont modifiables,
\item on peut combiner plusieurs types de données dans les valeurs, entiers, booléens, flottants, \dots
\item on peut ajouter des éléments à une extrémité pour un coût faible : \type{append},
\item on peut combiner des listes en les ajoutant,
\item on peut répéter une liste en la multipliant par un entier positif, 
\item on peut extraire une portion de liste : \type{liste[a : b]},
\item on peut itérer sur une liste ; \type{for x in liste:}, 
\item il existe de nombreuses méthodes de modification : \type{pop}, \type{remove}, \type{del}, \dots
\end{itemize}

Comme exemple de cette souplesse, il existe au moins 3 façons d'appliquer une fonction à une liste.

\begin{lstlisting}
def appliquer1(f, liste):
    n = len(liste)
    out = [0]*n
    for i in range(n):
        out[i] = f(liste[i])
    return out
\end{lstlisting}

\begin{lstlisting}
def appliquer2(f, liste):
    n = len(liste)
    out = []
    for i in range(n):
        out.append(f(liste[i]))
    return out
\end{lstlisting}

\begin{lstlisting}
def appliquer3(f, liste):
    return [f(x) for x in liste]
\end{lstlisting}
%--------------------------------------------------------------------------
\newpage
%--------------------------------------------------------------------------
\subsection{Inconvénients}
%--------------------------------------------------------------------------
%--------------------------------------------------------------------------
Cette richesse fonctionnelle ne va pas sans quelques  problèmes.

\begin{itemize}
\item Les méthodes  \type{pop}, \type{remove}, \type{del} et autres ont un coût caché qui rend leur utilisation dangereuse en raison même de leur simplicité apparente.
\item La richesse de la structure rend le temps de calcul sur des grandes listes plus lent que dans d'autres langages.
\item La création d'une liste neutre de $n$ éléments ne semble pas "naturelle".
\item On pourrait souhaiter que l'application d'une fonction à une liste s'écrive simplement \type{f(liste)}.
\end{itemize}

Cependant ce qui manque le plus pour un usage scientifique de Python est l'inaptitude des listes à représenter simplement des vecteurs. On aura souvent besoin de considérer les éléments d'une liste comme les composantes d'un vecteur auquel on voudrait appliquer les lois d'un espace vectoriel : addition de deux vecteurs et multiplication par un scalaire.
%--------------------------------------------------------------------------
%--------------------------------------------------------------------------
%--------------------------------------------------------------------------
\section{Le module Numpy}
%--------------------------------------------------------------------------
%--------------------------------------------------------------------------
%--------------------------------------------------------------------------
\subsection{Présentation}
%--------------------------------------------------------------------------
%--------------------------------------------------------------------------
Il existe un module qui permet d'utiliser des listes de manière plus efficace dans le cadre du calcul scientifique : {\tt  numpy}. C'est un ensemble de fonctions centrées autour d'un nouveau type de données, des tableaux nommés {\tt array}. 

Ce type présente des limitations par rapport aux listes de python :

\begin{itemize}
\item la longueur est fixée, on ne peut pas utiliser la méthode \type{append},
\item le type des composantes est unique et fixé à l'avance, tous les éléments seront soit des flottants, soit des entiers.
\end{itemize}

En contrepartie il y a des avantages :

\begin{itemize}
\item les tableaux peuvent être additionnés terme-à-terme comme des vecteurs,
\item on peut appliquer une fonction à un tableau, 
cela applique la fonction à tous les termes,
\item en particulier on peut multiplier un tableau par un scalaire, cela multiplie chaque composante,
\item les calculs sont plus rapides.
\end{itemize}

L'accès aux éléments, la longueur et l'extraction sont écrits de la même façon que dans le cas des listes python.
%--------------------------------------------------------------------------
%--------------------------------------------------------------------------
\subsection{Premières fonctions}
%--------------------------------------------------------------------------
%--------------------------------------------------------------------------
Le module doit être importé 
\begin{lstlisting}
import numpy as np
\end{lstlisting}
Il est recommandé de ne pas importer le module de manière anonyme (\type{from numpy import *}) et ne nom abrégé, \type{np}, est conventionnel.

Le module contient les fonctions mathématiques usuelles, il est préférable d'utiliser ces fonctions plutôt que celles du module \type{math}.
%--------------------------------------------------------------------------
\subsubsection{Création}
%--------------------------------------------------------------------------
Pour créer un tableau, plusieurs moyens sont possibles.

\begin{enumerate}
\item On peut convertir, par la fonction \type{np.array}, une liste python

\begin{lstlisting}
>>> m = np.array([1, 2, 5, 3])
>>> m
array([1, 2, 5, 3])
\end{lstlisting}

L'affichage par \type{print} supprime les virgules,
\begin{lstlisting}
>>> print(m)
[1 2 5 3]
\end{lstlisting}
\newpage
\item On peut créer un tableau de valeurs nulles avec {\tt np.zeros} :
\begin{lstlisting}
>>> np.zeros(4)
array([ 0.,  0.,  0.,  0.])
\end{lstlisting}
On remarquera que les valeurs sont des flottants.

\item On peut forcer le type des valeurs avec le paramètre optionnel \type{dtype}
\begin{lstlisting}
>>> np.zeros(4, dtype = int)
array([ 0,  0,  0,  0])
\end{lstlisting}
On peut choisir des types entiers spécifiques, signés/non signés, sur 8, 16, 32 ou 64 bits.

\item On peut créer un tableau de valeurs unitaires avec {\tt np.ones} :
\begin{lstlisting}
>>> np.ones(5)
array([1., 1., 1., 1., 1.])
\end{lstlisting}

\item On notera que le type des données est contraignant
\begin{lstlisting}
>>> m =  np.zeros(4, dtype = int)
>>> m[1] = 2.6
>>> m
array([0, 2, 0, 0])
\end{lstlisting}

Lorqu'un tableau est crée avec des valeurs entières alors qu'on veut faire des calculs flottants, on devra mettre au moins un des termes sous forme flottante.
\begin{lstlisting}
>>> m = np.array([1.0, 2, 5, 3])
>>> m
array([1., 2., 5., 3.])
\end{lstlisting}

\item La fonction \type{np.linspace(a, b, n)} crée un tableau de taille $n$ avec des valeurs de $a$ à $b$ également espacées.
\begin{lstlisting}
>>> np.linspace(0, 1, 5)
array([0.  , 0.25, 0.5 , 0.75, 1.  ])
\end{lstlisting}

\item La fonction \type{np.arange(a, b, r)} crée le tableau des valeurs qui commencent par $a$, espacées de $r$ et qui ne dépassent pas $b$ ($b$ est exclu).
\begin{lstlisting}
>>> np.arange(0, 1, 0.2)
array([0. , 0.2, 0.4, 0.6, 0.8])

>>> np.arange(0, 1, 0.3)
array([0. , 0.3, 0.6, 0.9])
\end{lstlisting}

\item La fonction \type{np.copy(a)} crée une copie indépendante de l'original.
\end{enumerate}
%--------------------------------------------------------------------------
\subsubsection{Opérations}
%--------------------------------------------------------------------------
On utilisera les tableaux \type{a = np.array([1.2, 2.0, -1.4])} et \type{b = np.array([2.0, 4, 3])}.
\begin{enumerate}
\item La somme de deux tableaux {\bf de même longueur} est le tableau des sommes.
\begin{lstlisting}
>>> a + b
array([3.2, 6. , 1.6])
\end{lstlisting}
La somme de deux tableaux de longueurs différentes engendre une erreur.
%--------------------------------------------------------------------------
\item Le produit de deux tableaux {\bf de même longueur} est le tableau des produit.
\begin{lstlisting}
>>> a*b
array([ 2.4,  8. , -4.2])
\end{lstlisting}
On peut aussi diviser un tableau par un autre.
%--------------------------------------------------------------------------
\item On peut ajouter un scalaire à un tableau, on obtient le tableau dans lequel le scalaire a été ajouté à chaque terme.
\begin{lstlisting}
>>> a + 2.7
array([3.9, 4.7, 1.3])
\end{lstlisting}
%--------------------------------------------------------------------------
\item On peut calculer le tableau des puissances des termes d'un tableau
\begin{lstlisting}
>>> a**2
array([1.44, 4.  , 1.96])
\end{lstlisting}
On peut donc appliquer un polynôme à un tableau
\begin{lstlisting}
def P(x):
    return x**3 - 2*x**2 + 7*x + 4

>>> P(a)
array([ 11.248,  18.   , -12.464])
\end{lstlisting}
%--------------------------------------------------------------------------
\item Plus généralement on peut appliquer une {\bf fonction \type{numpy}} à un tableau
\begin{lstlisting}
>>> np.sin(a)
array([ 0.93203909,  0.90929743, -0.98544973])
\end{lstlisting}
Cela est possible pour une fonction définie à partir des fonctions \type{numpy}.
\begin{lstlisting}
def f(x):
    return 1 + 2.5*np.sin(3*x) - np.cos(np.pi*x)

>>> f(a)
array([ 0.70271589, -0.69853875,  3.48795643])
\end{lstlisting}
%--------------------------------------------------------------------------
\item Une fonction définie avec des instructions conditionnelles ne s'applique pas directement ; elle n'est pas une fonction universelle (\Type{ufunc}) pour \type{numpy}. On peut la transformer pour qu'elle puisse agir sur les tableaux \type{numpy} avec l'instruction \type{np.vectorize}.
\begin{lstlisting}
def signe1(x):   
    if x > 0: 
        return 1    
    elif x == 0:
        return 0
    else:
        return -1

>>> signe1(a)
...
ValueError: ...

signe = np.vectorize(signe1)

>>> signe(a)
array([ 1,  1, -1])
\end{lstlisting} 
%--------------------------------------------------------------------------
\item On peut affecter une valeur à tous les termes d'une extraction :
{\bf on modifie alors le tableau}.
\begin{lstlisting}
>>> a[1:3] = 2.5
>>> a
array([1.2, 2.5, 2.5])
\end{lstlisting} 
%--------------------------------------------------------------------------
\end{enumerate}
%--------------------------------------------------------------------------
%--------------------------------------------------------------------------