%-------------------------------------------------------------------------------
%-------------------------------------------------------------------------------
%-------------------------------------------------------------------------------
\chapter{Réels}
%-------------------------------------------------------------------------------
%-------------------------------------------------------------------------------
\thispagestyle{empty}
%-------------------------------------------------------------------------------
%-------------------------------------------------------------------------------
On a vu que les entiers étaient représentés par leur écriture en base deux, formée de 0 et de 1.

On a vu aussi que les lettres étaient représentées par des entiers 

(la correspondance peut se voir à l'aide des fonctions \type{ord} et \type{chr}).

Les nombres utilisés par les sciences sont, le plus souvent, des réels. Il faut imaginer une représentation de ces nombres qui permette de les stocker en utilisant toujours le même nombre de bits afin de les manipuler plus simplement.

Cependant toute représentation des réels se heurte au problème de la finitude : les réels, comme les entiers, ne sont pas bornés et il faudra limiter les valeurs possibles. Mais il y a aussi une infinité de réels dans un intervalle et il faudra représenter les réels par des valeurs approchées : tous les réels dans un intervalle de la forme $]a-h;a+h[$ seront représenté par le même flottant $a$.

\medskip
Sur ces problèmes d'arrondi on pourra lire: 
\begin{lstlisting}
http://www-users.math.umn.edu/~arnold/disasters/patriot.html 
https://fr.wikipedia.org/wiki/Vol_501_d'Ariane_5 
\end{lstlisting}
%-------------------------------------------------------------------------------
%-------------------------------------------------------------------------------
%-------------------------------------------------------------------------------
\section{Représentations possibles sur 8 bits}
%-------------------------------------------------------------------------------
%-------------------------------------------------------------------------------
%-------------------------------------------------------------------------------

Nous allons étudier divers types de représentations possibles des réels et décrire leurs inconvénients.

Dans les exemples étudiés on se restreint à l'usage de 8 bits (un \Type{octet}) pour la représentation. Cette taille n'est pas réaliste pour un usage normal des flottants mais elle permet de visualiser ce qui se passe.
%-------------------------------------------------------------------------------
%-------------------------------------------------------------------------------
\subsection{Nombre à virgule fixe}
%-------------------------------------------------------------------------------
%-------------------------------------------------------------------------------
La première idée qui peut venir à l'esprit est de définir des approximations avec un nombre fixe de chiffres après la virgule. C'est ce qui est utilisé dans les tableurs.

Pour 8 bits on peut choisir

\begin{itemize}
   \item le premier bit indique le signe,
   \item il reste 7 bits qui permettent de représenter $2^7 = 128$ valeurs,
   \item les entiers sont divisés par 10.
 \end{itemize}
 
L'octet \begin{tabular}{|l|l|l|l|l|l|l|l|}
$b_0$&$b_1$&$b_2$&$b_3$&$b_4$&$b_5$&$b_6$&$b_7$\\ 
\hline \end{tabular}
($b_i\in\{0,1\}$) représente donc 
\[(-1)^{b_0}\frac{b_1+2b_2+4b_3+8b_4+16 b_5 +32 b_6 + 64 b_7}{10}\]

\begin{center}
\tikzpicture[scale=0.9]
\draw[->] (-0.5,0)  --  (13.5,0);
\foreach \n in {0,1,2,...,13}
  \draw (\n, -0.5) node[below] {$\n$} -- +(0,0.5);
\foreach \n in {0,1,2,...,11}
  \foreach \r in {0,1,2,...,9}
     \draw ({\n+\r/10},-0.1) -- +(0,0.2);
\foreach \r in {0,1,2,...,7}
  \draw ({12+\r/10},-0.1) -- +(0,0.2);
\endtikzpicture 
\end{center}

Ce format est en fait une représentation par des entiers avec un facteur d'échelle.

La précision est constante et il est très bien adapté dans le cas de calculs simples :
\begin{itemize}
  \item échanges entre différents composants,
  \item calculs avec une seule unité de mesure : finance, tailles standardisées, \dots
\end{itemize}

Les inconvénients sont nombreux

\begin{itemize}
    \item La précision est constante : on approche les grandes valeurs avec la même précision que les petites. On a le plus souvent besoin d'une précision relative.
    \item La multiplication fait vite sortir de l'ensemble des valeurs possibles.
    \item On ne peut pas gérer des très petites valeurs ou des très grandes valeurs.
  \end{itemize}
%-------------------------------------------------------------------------------
%-------------------------------------------------------------------------------
\subsection{Nombres rationnels}
%-------------------------------------------------------------------------------
%-------------------------------------------------------------------------------
On peut ne pas se limiter à des quotient à dénominateur fixe en codant par des nombres rationnels.

Pour 8 bits on peut choisir

\begin{itemize}
   \item le premier bit indiquer le signe,
   \item 4 bits pour le numérateur,
   \item 3 bits pour le dénominateur, on ajoutera 1 pour éviter un dénominateur nul.
 \end{itemize}

L'octet \begin{tabular}{|l|l|l|l|l|l|l|l|}
$b_0$&$b_1$&$b_2$&$b_3$&$b_4$&$b_5$&$b_6$&$b_7$\\ 
\hline \end{tabular}
($b_i\in\{0,1\}$) représente donc 
\[(-1)^{b_0}\frac{b_1+2b_2+4b_3+8b_4}{1 + b_5 + 2 b_6 + 4 b_7}\]

\begin{center}
\tikzpicture[scale=0.8]
\draw[->] (-0.5,0)  --  (15.5,0);
\foreach \n in {1,2,...,15}
  {\draw (\n, -0.5) node[below] {$\n$} -- +(0,0.7);
   \foreach \r in {2, 3,...,8}
     \draw ({\n/\r},-0.1) -- +(0,0.2);};
\draw (0, -0.5) node[below] {$0$} -- +(0,0.7);
\endtikzpicture 
\end{center}

Ce format est bien adapté à certains calculs algébriques

Mais il conserve de nombreux inconvénients.

\begin{itemize}
    \item Les valeurs sont très irrégulièrement espacées.
    \item Il y a de nombreux doublons ($\frac 21$, $\frac 42$, $\frac 63$, \dots)
    \item Les calculs des fonctions usuelles (sin, cos, ln, exp, \dots) ne sont pas aisés.
  \end{itemize}
%-------------------------------------------------------------------------------
%-------------------------------------------------------------------------------
\subsection{Nombres à virgule flottante}
%-------------------------------------------------------------------------------
%-------------------------------------------------------------------------------
On peut enfin utiliser une virgule flottante : au lieu de considérer des entiers divisés par une constante, on considère les nombres de la forme $\pm n.b^k$ avec $n$ entier strictement positif, $b$ fixé (la base) et $k$ entier relatif, l'exposant réel.

Dans la suite on ne considérera que la base 2 : les nombres sont de la forme $\pm n.2^k$. Les nombres représentés seront toujours des fractions de la forme $\frac{n}{2^s}$.

On utilisera donc
\begin{itemize}
    \item un bit pour le signe
    \item $p$ bits pour $n$
    \item $q$ bits pour $k$
    \item avec $1+p+q$ égal au nombre de bits utilisés pour la représentation.
\end{itemize}

Les $p$ bits permettent de définir $2^p$ valeurs possibles pour $n$.

On remarque qu'il peut exister plusieurs représentations pour un même nombre : par exemple $1,75$ est représenté par $7.2^{-2}$ mais aussi par $14.2^{-3}$, $28.2^{-4}$ \dots

Pour obtenir une représentation unique on va imposer $n$ compris entre $2^p$ et $2^{p+1}-1$, il y a bien $2^p$ valeurs.

Ainsi les $p$ bits définissent un entier non signé $N_1$, $q$ autres bits définissent un entier $N_2$ et cela se traduit en le réel $\displaystyle (1)^s \bigl(2^p+N_1\bigr).2^{N_2-h}
=\left(1+\frac{N_1}{2^p}\right)2^{N_2-h+p}$.

Le problème le plus immédiat est qu'on ne peut pas représenter 0,  on verra dans la partie suivante une correction possible.

\newpage

Pour 8 bits on peut choisir

\begin{itemize}
   \item le premier bit indiquer le signe,
   \item $p=3$ bits pour la mantisse.
   \item $q=4$ bits pour l'exposant, que l'on décale en retranchant 7 pour obtenir des exposants entre $-7$ et 8.
 \end{itemize}



L'octet \begin{tabular}{|l|l|l|l|l|l|l|l|}
$b_0$&$b_1$&$b_2$&$b_3$&$b_4$&$b_5$&$b_6$&$b_7$\\ 
\hline \end{tabular}
($b_i\in\{0,1\}$) représente donc 
\[(-1)^{b_0}\left(1 + \frac m{8}\right)2^{e-7}\text{ avec }m = b_7+2b_6+4b_5\text{ et } e=b_4+2b_3+4b_2+8b_1\]

{\bf Exemples}
\begin{enumerate}
  \item Le plus petit nombre positif représentable est $\left(1 + \frac 0{8}\right)2^{0-7} = 0,0078125$.
  \item Le suivant est $\left(1 + \frac 1{8}\right)2^{0-7} \simeq 0,00879$.  
  \item Le grand petit nombre représentable est $\left(1 + \frac 7{8}\right)2^{15-7} = 480$.
  \item \type{01001011} donne $e = 9$ et $m = 3$ donc représente $\left(1 + \frac {3}{8}\right)2^{9-7} = 5,5$.
  \item Pour représenter $\pi$ on commence par se placer entre 1 et 2 donc on considère $\frac \pi2$.
  
On veut $\frac \pi2$ proche de $1+ \frac m{8}$ donc $m$ entier proche de $8\bigl(\frac \pi2 - 1)= 4,56...$.
  
  {\bf On choisit d'arrondir au plus proche.}\footnote{Pour approcher un nombre de la forme $\frac n2$, on choisira l'entier pair le plus proche.}
  
  On approche donc $\pi$ par $\left(1 + \frac 4{8}\right)2^{1} = 3, 0$. 
  
  On a $m = 3$ et $e=8$ donc $\pi$ est représenté par \type{01000011}.
  \item 17 n'est pas représentable, il est approché par 16.
  \item Pour représenter 300, on approche $\frac{300}{256}$ par une fraction de la forme $1 + \frac m8$ donc $m$ proche de $\frac{11}8$ : on choisit $m=1$ et 300 est approché par $\bigl(1+\frac 18\bigr)2^8=286$ qui est représenté par \type{01111001}.
\end{enumerate}
\begin{figure}[h]
  \begin{center}
    \tikzpicture[xscale=24]
    \draw (0,0)  --  (0.5,0);
    \foreach \e in {1, 2, 4, 8, 16, 32}
      \foreach \m in {0, 1, 2, ..., 7}
        \draw ({(1+\m/8)*\e/128},0) -- +(0,0.2);
    \foreach \k in {0, 0.1, 0.2, 0.3, 0.4, 0.5} 
      \draw (\k, -0.5) node[below] {$\k$} -- +(0,0.5);
    \endtikzpicture 
    \caption{Nombres représentables en deça de 0.5}
    \label{zero_un}
  \end{center}
\end{figure}

\begin{figure}[h]
  \begin{center}
    \tikzpicture[xscale=0.75]
    \draw (0,0)  --  (15.5,0);
    \foreach \e in {1, 2, 4, 8, 16, 32}
      \foreach \m in {0, 1, 2, ..., 7}
        \draw ({(1+\m/8)*\e/4}, 0) -- +(0, 0.2);
    \foreach \k in {0,1,2,3,...,15} 
      \draw (\k, -0.5) node[below] {$\k$} -- +(0,0.5);
    \endtikzpicture 
    \caption{Nombres représentables entre $0.25$ et 15}
  \end{center}
\end{figure}

\begin{figure}[h]
  \begin{center}
    \tikzpicture[xscale=3/128]
    \draw (0,0)  --  (512,0);
    \foreach \e in {1, 2, 4, 8, 16, 32}
      \foreach \m in {0, 1, 2, ..., 7}
        \draw ({(1+\m/8)*\e*8}, 0) -- +(0, 0.2);
    \foreach \k in {0,50, 100, ..., 500} 
      \draw (\k, -0.5) node[below] {$\k$} -- +(0,0.5);
    \endtikzpicture 
    \caption{Nombres représentables au dela de 10}
  \end{center}
\end{figure}

%-------------------------------------------------------------------------------
%-------------------------------------------------------------------------------
\subsection{Dénormalisation}
%-------------------------------------------------------------------------------
%-------------------------------------------------------------------------------
Les nombres à virgule flottante permettent une meilleure répartition des nombres représentables : il y a le même nombre de valeurs entre $a$ et $2a$ pour tout $a$. On parvient ainsi à une meilleure étendue des valeurs pour un nombre de bits déterminé.

Cependant la figure \ref{zero_un} montre un problème important : il n'y a pas de réel représentable autour de 0 alors que la précision est bonne au delà du premier réel représenté.

Pour obtenir un meilleur résultat on choisit de {\bf dénormaliser} les nombres représentés lorsque l'exposant $e$ est nul. 

Dans le cas d'une représentation avec 8 bits on remplace $(-1)^{b_0}\left(1 + \frac m{8}\right)2^{0-7}$ par $(-1)^{b_0}\frac m{8}2^{0-7+1}$.

La précision est moindre pour les petits flottants mais on approche des nombres autour de 0.

\begin{figure}[h]
  \begin{center}
    \tikzpicture[xscale=48]
    \draw (0,0)  --  (0.25,0);
    \foreach \e in {2, 4, 8, 16}
      \foreach \m in {0, 1, 2, ..., 7}
        \draw ({(1+\m/8)*\e/128},0) -- +(0,0.2);
    \foreach \m in {0, 1, 2, ..., 7}
      \draw ({\m/8/64},0) -- +(0,0.2);
    \foreach \k in {0, 0.03125, 0.0625, 0.125, 0.25} 
      \draw (\k, -0.5) node[below] {$\k$} -- +(0,0.5);
    \endtikzpicture 
    \caption{Représentables dénormalisés}
  \end{center}
\end{figure}


\begin{enumerate}
  \item Le plus petit nombre positif représentable est maintenant $ \frac 1{16}2^{-2} = 0,015625$.
  \item $0,015625$ est le pas pour les premiers flottants.  
  \item Les nombres représentables au-delà de $0,25$ sont les mêmes.
\end{enumerate}

Pour des calculs scientifiques, cette représentation est celle qui est choisie.

Elle a de nombreux avantages mais il reste des problèmes qui rendent l'interprétation des calculs parfois difficiles. On verra dans la suite quelques-uns de ces problèmes.
%-------------------------------------------------------------------------------
%-------------------------------------------------------------------------------
\section{Les nombres flottants dans Python}
%-------------------------------------------------------------------------------
%-------------------------------------------------------------------------------
Les langages modernes utilisent des réels définis en virgule flottante sur 32 bits (simple précision) ou sur 64 bits (double précision) en respectant un standard mondial : la norme {\sc ieee}-754.

Python, dans la version 3, n'utilise que des flottants sur 64 bits.

Dans cette représentation les bits de l'exposant sont placés avant ceux de la mantisse.
%-------------------------------------------------------------------------------
%-------------------------------------------------------------------------------
\subsection{Les normes IEEE-754}
%-------------------------------------------------------------------------------
%-------------------------------------------------------------------------------
La norme est définie, pour 32 bits, par
%-------------------------------------------------------------------------------
\begin{itemize}
\item un bit pour le signe, 
\item 8 bits pour l'exposant donc $e$ est compris entre 0 et $255$,
\item le décalage est $2^{7}-1=127$ et les exposant 0 et $255$ ne sont pas utilisés, ils sont réservés pour des nombres dénormalisés,
\item il reste 23 bits pour la mantisse.
\end{itemize}
%-------------------------------------------------------------------------------
\medskip

Les 4 octets \begin{tabular}{|l|l|l|l|l|}
$b_0$&$b_1$&$\cdots$&$b_{30}$&$b_{31}$\\ 
\hline \end{tabular} représentent donc 
\[(-1)^{b_0}\left(1 + \frac m{2^{23}}\right)2^{e-127}\text{ avec }m = \sum_{k=9}^{31} b_k.2^{31-k}\text{ et } e=\sum_{k=1}^{8} b_k.2^{8-k}\]

\medskip

La norme est définie, pour 64 bits, par
%-------------------------------------------------------------------------------
\begin{itemize}
\item un bit pour le signe, 
\item 11 bits pour l'exposant donc $e$ est compris entre 0 et $2047$,
\item le décalage est $2^{10}-1=1023$ et les exposants 0 et $2047$ ne sont pas utilisés, ils sont réservés pour des nombres dénormalisés,
\item il reste 52 bits pour la mantisse.
\end{itemize}
%-------------------------------------------------------------------------------
\medskip

Les 8 octets \begin{tabular}{|l|l|l|l|l|}
$b_0$&$b_1$&$\cdots$&$b_{62}$&$b_{63}$\\ 
\hline \end{tabular} représentent donc 
\[(-1)^{b_0}\left(1 + \frac m{2^{52}}\right)2^{e-1023}\text{ avec }m = \sum_{k=12}^{63} b_k.2^{63-k}\text{ et } e=\sum_{k=1}^{11} b_k.2^{11-k}\]
%-------------------------------------------------------------------------------
%-------------------------------------------------------------------------------
\subsection{Exemples}
%-------------------------------------------------------------------------------
%-------------------------------------------------------------------------------
Pour coder  $11.0$, on l'écrit 
\[11.0 = \frac{11}8.2^3  = \frac{8+3}8.2^{1026-1023} = \left(1 + \frac{3.2^{49}}{2^{52}}\right).2^{1026-1023}\] 

avec $3 = 2^0+2^1$ et $1026 = 2^1 + 2^{10}$ donc le codage sur 64 bits est 
%-------------------------------------------------------------------------------
\begin{center}
$\underbrace{\begin{array}[t]{|l|}
0 \\ 
\hline 
\end{array}}_{\text{signe}}$
$\underbrace{\begin{array}[t]{|l|l|l|l|l|l|l|l|l|l|l|}
1&0&0&0&0&0&0&0&0&1&0 \\ 
\hline 
\end{array}}_{\text{1023+exposant}}$
$\underbrace{\begin{array}[t]{|l|l|l|l|l|l|l|l|l|l|l|}
0&0&0&0&0&\dots&0&0&0&1&1 \\ 
\hline 
\end{array}}_{\text{mantisse sur 52 bits}}$\\
\end{center}
%-------------------------------------------------------------------------------
Pour coder le nombre d'Avogadro $N = 6.02 10^{23}$, 
\begin{enumerate}
    \item $2^{78} \le N < 2^{79}$
    \item $N.2^{-78} = 1.9918509150276906$
    \item $0.9918509150276906*2^{52} = 4466899411325793$ 
    \item $4466899411325793$ en vase 2 s'écrit
     
     \type{1111110111101001111100010000101010001101001101100001}

     ce qui donne les 52 derniers bits
     \item le premier est 0, le signe est positif
     \item les 12 suivants codent 75+1023 : \type{10001001101}
\end{enumerate}
%-------------------------------------------------------------------------------
\begin{lstlisting}
>>> (1+4466899411325793/2**52)*2**78
6.02e+23
\end{lstlisting}
%-------------------------------------------------------------------------------
\medskip

Voici quelques valeurs obtenues avec la représentation sur 64 bits
\begin{itemize}
    \item L'exposant $e=1$ permet d'obtenir la valeur $\displaystyle \left(1+\frac 0{2^{52}}\right)2^{-1023}\sim 10^{-308}$
    \item Les nombres dénormalisés sont de la forme $\displaystyle \frac k{2^{52}}.2^{-1023}$. Le plus petit flottant dénormalisé strictement positif est donc $2^{-1075}\sim 10^{-324}$
    \item Le plus grand réel représentable est $\displaystyle \left(1+\frac {2^{52}-1}{2^{52}}\right)2^{1023}\sim 10^{308}$, 
    
    le précédent diffère de $2^{971}\sim.10^{292}$.
    \item La précision pour les réels entre 1 et 2 est de $2^{-52}\sim 2.10{-16}$.
    \item Le plus petit entier qui n'est pas représenté sans erreur est $2^{53}+1$.
%-------------------------------------------------------------------------------
\begin{lstlisting}
>>> 2**53+1
9007199254740993
>>> float(2**53+1)
9007199254740992.0
\end{lstlisting}
%-------------------------------------------------------------------------------
\end{itemize}
%-------------------------------------------------------------------------------
%-------------------------------------------------------------------------------
\subsection{Problèmes d'arrondi}
%-------------------------------------------------------------------------------

%-------------------------------------------------------------------------------
\subsubsection{Comparaison à zéro}
%-------------------------------------------------------------------------------
%-------------------------------------------------------------------------------
Deux nombre trop proches l'un de l'autre seront considérés comme égaux.
%-------------------------------------------------------------------------------
\begin{lstlisting}
>>> i=1
>>> while 1.0+10**(-i)>1.:
...     i+=1
>>> i
16
\end{lstlisting}
%-------------------------------------------------------------------------------
Cela signifie qu'une variation relative de $10^{-16}$ n'est par perceptible.

%-------------------------------------------------------------------------------
%-------------------------------------------------------------------------------
\begin{Exercise}[title={ Equation de degré 2}, label=edd]
Ecrire une fonction \type{racine(a,b,c)} qui donne le nombre de racines de l'équation $ax^{2}+bx+c=0$.

La tester pour $a = 0.1$, $b = 0.6$, $c = 0.9 $.
\end{Exercise}

Le résultat de la soustraction de deux nombres a et b proches peut être de l'ordre de l'imprécision des calculs.

On a vu dans le paragraphe précédent que l'incertitude semble de l'ordre de $10^{-16}$ pour un nombre de l'ordre de $1$ donc si on soustrait des nombres de l'ordre de $10^{16}$ on devrait avoir une imprécision de l'ordre de $1$.
%-------------------------------------------------------------------------------
\newpage
%-------------------------------------------------------------------------------
\subsubsection{Sommes et ordre de sommation}
%-------------------------------------------------------------------------------
%-------------------------------------------------------------------------------
Si on doit ajouter $3$ nombres, on commencera par ajouter les deux plus petits puis on ajoutera le troisième, l'erreur générée sera alors moins grande car elle est proportionnelle au plus grand des termes de chaque addition.\\
Par exemple, on sait que $\sum\limits_{k=1}^{n}\frac{1}{k^4}$ a pour limite $\frac{\pi^{4}}{90}$.\\
Si on opère dans l'ordre, on commence par ajouter les plus grands termes, si on opère en partant de la fin, on commence par ajouter les plus petits.

Voici ce que donne le calcul:
%-------------------------------------------------------------------------------
\begin{lstlisting}
from math import pi
def zeta(n):
    l=[1/i**4 for i in range(1,n+1)]
    somme=0
    for i in range(n):
        somme=somme+l[i]
    sommeinverse=0
    for i in range(n):
        sommeinverse=sommeinverse+l[-i-1]
    return somme-pi**4/90,sommeinverse-pi**4/90

print(zeta(66))
print(zeta(2**18))
\end{lstlisting}
%-------------------------------------------------------------------------------
\begin{lstlisting}
(-1.1333517326850284e-06,-1.1333517324629838e-06)
(-2.7688962234151404e-13,2.220446049250313e-16)
\end{lstlisting}
%-------------------------------------------------------------------------------
%-------------------------------------------------------------------------------
\subsection{Le module fractions}
%-------------------------------------------------------------------------------
%-------------------------------------------------------------------------------
Si on utilise le module \type{fractions} on peut travailler avec des valeurs fractionnaires des nombres et éviter beaucoup des problèmes précédents. Le prix à payer en termes d'efficacité est par contre élevé et de toutes façons, ça ne règle pas le problème de la représentation des nombres irrationnels.
%-------------------------------------------------------------------------------
\begin{lstlisting}
from fractions import Fraction
a=Fraction(16,-10) 
\end{lstlisting}
%-------------------------------------------------------------------------------
Le module $decimal$ permet aussi de travailler avec des nombres décimaux.
%-------------------------------------------------------------------------------
\begin{lstlisting}
from decimal import Decimal
b=Decimal('1.1') #noter l usage des guillemets
\end{lstlisting}
%-------------------------------------------------------------------------------
On peut aussi choisir la précision (le nombre maximal de décimales).
%-------------------------------------------------------------------------------
\begin{lstlisting}
from decimal import getcontext
getcontext().prec=100
\end{lstlisting}
%-------------------------------------------------------------------------------
Forcément, choisir $100$ décimales rend les calculs moins rapides.
%-------------------------------------------------------------------------------
%-------------------------------------------------------------------------------
