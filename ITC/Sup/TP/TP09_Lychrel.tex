%-------------------------------------------------------------------------------
%-------------------------------------------------------------------------------
%-------------------------------------------------------------------------------
\chapter{Jeux sur les chiffres}
%-------------------------------------------------------------------------------
%-------------------------------------------------------------------------------
\thispagestyle{empty}
%-------------------------------------------------------------------------------
%-------------------------------------------------------------------------------

\vskip -2cm

%-------------------------------------------------------------------------------
%-------------------------------------------------------------------------------
%-------------------------------------------------------------------------------
\section{Retournement} 
%-------------------------------------------------------------------------------
%-------------------------------------------------------------------------------
%-------------------------------------------------------------------------------
Dans cette partie, on étudie le retournement d'un nombre : par exemple 147 devient 741.

Pour retourner un nombre, on propose l'algorithme suivant :
\begin{enumerate}
    \item on calcule la liste de ses chiffres, par exemple 147 donne \type{[1, 4, 7]},
    \item on retourne cette liste, on obtient ici,  \type{[7, 4, 1]}, 
    \item on revient à l'entier associé à la liste $(7*10 + 4)*10 + 1 = 741$
\end{enumerate}

La clé pour le calcul des chiffres est de remarquer que le chiffre des unités de $n$ est \type{n\%10} et que les autres chiffres sont obtenus à partir de \type{n//10} ;  on recommence ensuite.

Dans le cas de 741, 
\begin{itemize}
    \item le chiffre des unités est \type{1 = 741\%10} et les autres chiffres proviennent de \type{74 = 741//10}
   \item le chiffre des dizaines est \type{4 = 74\%10} et les chiffres suvants se calculent avec \type{7 = 74//10}
   \item le chiffre des centaines est \type{7 = 7\%10} et il reste \type{0 = 7//10} donc on s'arrête.
\end{itemize}

On remarque que les chiffres sont calculés "à l'envers" ; on doit déterminer leur nombre afin de définir un tableau de taille adaptée et le remplir de gauche à droite.

\medskip

Pour calculer le nombre de chiffres de $n$, on remarque que si $n$ admet $p$ chiffres alors \type{n//10} en admet $p-1$ ; on divise donc par 10 tant qu'on peut, c'est-à-dire tant que $n$ est non nul en ajoutant 1 à chaque étape au nombre de chiffres.

On notera qu'il est cohérent de considérer que 0 s'écrit avec 0 chiffre.
%-------------------------------------------------------------------------------
%-------------------------------------------------------------------------------
\begin{Exercise}[title= Nombre de chiffres]\it
Écrire une fonction \type{nb\_chiffres(n)} qui calcule le nombre de chiffres de $n$.

On supposera que $n$, sans avoir besoin de le prouver, est un entier positif.
\end{Exercise}
%-------------------------------------------------------------------------------
\begin{Answer} 
\begin{lstlisting}
def nb_chiffres(n):
    p = 0
    while n > 0:
        p = p + 1
        n = n//10
    return p
\end{lstlisting}
\end{Answer}
%-------------------------------------------------------------------------------
%-------------------------------------------------------------------------------

Pour calculer les chiffres de $n$, on a vu qu'on remplissait une liste à partir de la droite ; si $p$ est le nombre de chiffres, on calcule d'abord le chiffre des unités que l'on place à la position $p-1$, puis on calcule le chiffre à la position $p-2$ jusqu'à 0.
%-------------------------------------------------------------------------------
%-------------------------------------------------------------------------------
\begin{Exercise}[title= Chiffres]\it
Écrire une fonction \type{chiffres(n)} qui calcule les chiffres de $n$ sous forme d'une liste.
\end{Exercise}
%-------------------------------------------------------------------------------
\begin{Answer} 
\begin{lstlisting}
def chiffres(n):
    p = nb_chiffres(n)
    chf = [0]*p
    for i in range(p):
        c = n%10 
        n = n//10
        chf[p-1-i] = c
    return chf
\end{lstlisting}
\end{Answer}
%-------------------------------------------------------------------------------
%-------------------------------------------------------------------------------
On a besoin de convertir une liste de chiffres en un nombre.

La méthode suggéré ci-dessus est la méthode de Hörner qui minimise les calculs.

On peut l'écrire de la façon suivante.
%-------------------------------------------------------------------------------
\begin{lstlisting}
def nombre(liste):
    p = len(liste)
    n = 0
    for i in range(p):
        n = 10*n + liste[i]
    return n
\end{lstlisting}
%-------------------------------------------------------------------------------
On pourra vérifier qu'on a bien \type{nombre(chiffres(n)) = n} pour quelques valeurs de $n$.

\smallskip

Pour retourner une liste, on pourra utiliser
\begin{lstlisting}
etsil = liste[ : : -1]
\end{lstlisting}
%-------------------------------------------------------------------------------
%-------------------------------------------------------------------------------
\begin{Exercise}[title= Retournement d'un entier]\it
Écrire une fonction \type{retournement(n)} qui le nombre obtenu par retournement de $n$.
\end{Exercise}
%-------------------------------------------------------------------------------
\begin{Answer} 
\begin{lstlisting}
def retournement(n):
    liste = chiffres(n)
    etsil = liste[ : : -1]
    return nombre(etsil)
\end{lstlisting}
\end{Answer}
%-------------------------------------------------------------------------------
%--------------------------------------------------------------------------
%--------------------------------------------------------------------------
\section{Nombres de Lychrel}
%--------------------------------------------------------------------------
%--------------------------------------------------------------------------
%--------------------------------------------------------------------------
Un nombre est un palindrome s'il est égal à son nombre retourné.
%--------------------------------------------------------------------------
%--------------------------------------------------------------------------
\begin{Exercise}[title= Test de palindrome]\it 
Écrire une fonction \type{test(n)} qui renvoie  \type{True} ou \type{False} selon que $n$ est un palindrome ou non.
\end{Exercise}
%--------------------------------------------------------------------------
\begin{Answer}
\begin{lstlisting}
def test(n):
    m = retournement(n)
    return n == m
\end{lstlisting}
\end{Answer}
%--------------------------------------------------------------------------
%--------------------------------------------------------------------------

On définit une suite $(u_n)$ par son premier terme, $u_0$, et par la relation de récurrence :

$u_{n+1}$ est la somme de $u_n$ et du miroir de $u_n$.

Par exemple si on pose $u_0=79$ alors $u_1=79+97=176$, $u_2 = 176+671=847$, $u_3 = 847+748=1595$, $u_4=1595+5951=7546$, $u_5=7546+6457=14003$ et $u_6=14003+30041=44044$.

On remarque que $u_6$ est un palindrome.

%--------------------------------------------------------------------------
%--------------------------------------------------------------------------
\begin{Exercise}[title= Calcul des termes de la suite]\it 
Écrire une fonction \type{suite(u0, n)} qui calcule la valeur de $u_n$ pour la suite de premier terme $u_0$ définie ci-dessus.
\end{Exercise}
%--------------------------------------------------------------------------
\begin{Answer}
\begin{lstlisting}
def suite(u0, n):
    u = u0
    for i in range(n):
        u = u + retournement(u)
    return u
\end{lstlisting}
\newpage
\end{Answer}
%--------------------------------------------------------------------------
%--------------------------------------------------------------------------
Si on part de $u_0=28$ alors $u_1=28+82=110$, $u_2=110+11=121$ qui est aussi un palindrome.

En fait la suite parvient souvent à un nombre palindrome rapidement : les nombres qui ne semblent pas aboutir à un palindrome sont appelés nombres de {\bf Lychrel}. Le plus petit nombre de Lychrel est 196. 

On admettra que pour $a < 10000$, $an$ est un nombre de Lychrel si et seulement si aucun des termes $u_n$ pour $0\le n < 25$ n'est un palindrome.
%--------------------------------------------------------------------------
%--------------------------------------------------------------------------
\begin{Exercise}\it Combien y-a-t-il de nombres de Lychrel inférieurs à 1000 ?

On affichera ces nombres à l'écran.
\end{Exercise}
%--------------------------------------------------------------------------
\begin{Answer}
\begin{lstlisting}
nb = 0
for a in range(1, 1000):
    u = a
    n = 0
    while n < 50 and not test(u):
        u = u + retournement(u)
        n = n + 1
    if n == 50:
        print(a)
        nb = nb + 1
print(nb)
\end{lstlisting}
Il y en a 13.
\end{Answer}
%--------------------------------------------------------------------------
%--------------------------------------------------------------------------
\begin{Exercise}\it Déterminer le premier entier $a$ tel que la suite $(u_n)$ de premier terme $a$ n'aboutit à un palindrome que pour $u_{24}$. On admettra que cela advient entre 1  et 195.

On donnera aussi la valeur de $u_{24}$.
\end{Exercise}
%--------------------------------------------------------------------------
\begin{Answer}
\begin{lstlisting}
for a in range(1, 196):
    u = a
    n = 0
    while not test(u):
        u = u + retournement(u)
        n = n + 1
    if n == 24:
        print(a)
        print(u)
        break
\end{lstlisting}
Pour $a = 89$, $u_{24} = 8813200023188$
\end{Answer}
%--------------------------------------------------------------------------
%--------------------------------------------------------------------------
%--------------------------------------------------------------------------
\section{Exercices tirés du projet Euler}
%------------------------------------------------------------
%------------------------------------------------------------
\begin{Exercise}[title={Projet Euler 20}]\it
Que vaut la somme des chiffres de 100! ?

Python manie les entiers sans limitation.
\end{Exercise}
%------------------------------------------------------------
\begin{Answer}
\begin{lstlisting}
def facto(n):
    f = 1
    for i in range(n):
        f = f*(i+1)
    return f
    
l = chiffres(facto(100))
somme = 0
for x in l:
    somme = somme + x
print("La somme des chiffres de 100! est",somme)
\end{lstlisting}
648
\end{Answer}
%--------------------------------------------------------------------------
%--------------------------------------------------------------------------
On peut remarquer que $145 = !1+!4+!5$. Il existe un second entier vérifiant cette propriété d'être la somme des factorielles de ses chiffres (en dehors des cas triviaux $1 = !1$ et $2=!2$).
%--------------------------------------------------------------------------
%--------------------------------------------------------------------------
\begin{Exercise}[title = {Projet Euler 34}]\it 
Déterminer l'entier $n> 145$ qui est la somme des factorielles de ses chiffres.
\end{Exercise}
%--------------------------------------------------------------------------
\begin{Answer}
\begin{lstlisting}
fact = [1, 1, 2, 6, 24, 120, 720, 5040, 40320, 362880]
for n in range(10000, 100000):
    l = chiffres(n)
    s = 0
    for x in l:
        s = s + fact[x]
    if n ==  s:
        print(n)
        break
\end{lstlisting}
40585
\end{Answer}
%--------------------------------------------------------------------------
%--------------------------------------------------------------------------
\begin{Exercise}[title = {Projet Euler 56}]\it 
Quelle est la somme des chiffres maximale pour les entiers de la forme $a^b$ avec $a$ et $b$ entiers, $a< 100$ et $b < 100$. 
\end{Exercise}
%--------------------------------------------------------------------------
\begin{Answer}
\begin{lstlisting}
max = 1
a_max = 1
b_max = 1
for a in range(100):
    for b in range(101):
        s = sum_c(a**b)
        if s > max:
            print(a, b, s)
            max = s
            a_max = a
            b_max = b
print(a_max, b_max, max)
\end{lstlisting}
La somme est 972 pour $99^{95}$.
\end{Answer}
%--------------------------------------------------------------------------
%--------------------------------------------------------------------------
