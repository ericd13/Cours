%-------------------------------------------------------------------------------
%-------------------------------------------------------------------------------
\chapter{Variables}
%-------------------------------------------------------------------------------
%-------------------------------------------------------------------------------
\thispagestyle{empty}
%-------------------------------------------------------------------------------
%-------------------------------------------------------------------------------
\begin{abstract} Dans ce T.P. nous allons utiliser l'affectation et les bénéfices de l'usage des variables.
\end{abstract}

\bigskip


{\bf Quelques rappels et conseils.}
\begin{itemize}
    \item Donnez à vos variables des noms parlants qui permettent de comprendre leur rôle. Ces noms ne doivent contenir que des lettres et des chiffres (pas de ponctuation, pas d'espace). Le caractère \type{\_} (underscore) est aussi autorisé.
    \item Il pourra parfois être utile de stocker aussi des résultats de calcul intermédiaires en créant d'autres variables.
    \item La multiplication implicite (par exemple $ab$ pour dire $a\times b$) n'est pas comprise par Python, donc utilisez forcément le symbole \type{*} pour les multiplications.
    \item La racine  carrée peut s'écrire \type{**0.5}
    \item Les fonctions et constantes mathématiquesne sont pas disponibles par défaut. Ajouter la ligne \type{import math} au début de votre code. Vous disposerez alors de \type{math.pi} pour la valeur numérique de $\pi$, \type{math.sin} pour la fonction sinus \dots.
    \item Dans tous les exercices comportant des valeurs numériques, ne les utilisez jamais directement dans les calculs. Cela rend la relecture du code pénible (et donc les erreurs moins faciles à détecter) et si une valeur apparaît plusieurs fois vous allez la retaper inutilement. Stockez les valeurs numériques dans des variables avec la syntaxe \type{nom\_variable = valeur}.
    \item L'ordinateur ne gère pas les unités, donc rentrez les valeurs numériques directement dans les bonnes unités,  celles fournies conviennent le plus souvent.
    \item les puissances de 10 ont une écriture spécifique : \SI{4.21e-11} s'écrit \type{4.21e-11} (il faut que l'exposant soit entier et que les valeurs soient explicitement numériques.
\end{itemize}
%-------------------------------------------------------------------------------
%-------------------------------------------------------------------------------
\section{Opérations simples}
%-------------------------------------------------------------------------------
%-------------------------------------------------------------------------------
\begin{Exercise}
{\it 
Voici un petit programme :
\begin{lstlisting}
x = 10
y = 15
z = x + y
x = y
y = z
print(x + y + z)
\end{lstlisting}

Que va-t-il donner comme résultat ? Vérifier en l'exécutant depuis l'éditeur.
}
\end{Exercise}
%-------------------------------------------------------------------------------
%-------------------------------------------------------------------------------
\begin{Exercise}
{\it Que voudrait-on faire avec les lignes suivantes ?
\begin{lstlisting}
x = 10
y = 15
y = x
x = y
\end{lstlisting}

Obtient-on le résultat souhaité ? Proposer une manière d'échanger les valeurs de deux variables.}
\end{Exercise}
%-------------------------------------------------------------------------------
%-------------------------------------------------------------------------------
\section{Calculs physiques}
%-------------------------------------------------------------------------------
%-------------------------------------------------------------------------------
\subsection{Deuxième équivalence}
%-------------------------------------------------------------------------------
%-------------------------------------------------------------------------------
On dose l'acide oxalique (un diacide) dont la concentration $C$ est inconnue avec un volume initial $V_0=\SI{20e-3}{L}$) par de la soude de concentration $C_b=\SI{5e-2}{mol.L^{-1}}$. La première équivalence est peu visible, donc on ne peut exploiter que la seconde, pour laquelle :
	\begin{align*}
	2CV_0=C_bV_\text{éq2}
	\quad\text{avec}\quad
	V_\text{éq2}=\SI{12e-3}{L}
	\end{align*}
%-------------------------------------------------------------------------------
%-------------------------------------------------------------------------------
\begin{Exercise}
{\it Calculez numériquement $C$.}
\end{Exercise}
%-------------------------------------------------------------------------------
\begin{Answer}
\begin{lstlisting}
V0 = 20e-3
Veq2 = 12e-3
Cb = 5e-2
C = Cb*Veq2/(2*V0)
print(C)
\end{lstlisting}
On trouve $C=\SI{1,5e-2}{mol.L^{-1}}$.
\end{Answer} 
%-------------------------------------------------------------------------------
%-------------------------------------------------------------------------------
\subsection{Pulsation plasma}
%-------------------------------------------------------------------------------
%-------------------------------------------------------------------------------
À quelques dizaines de kilomètres d'altitude se trouve une couche d'atmosphère appelée \emph{ionosphère}. Celle-ci est dans un état de la matière appelée \emph{plasma} et ne laisse passer les ondes télécom que si leur fréquence est assez élevée. On donne :
\begin{align*}
    \begin{dcases}
        \nu=\frac{\omega}{2\pi} & \text{fréquence}\\
        k^2=\frac{\omega^2-\omega_p^2}{c^2} & \text{nombre d'onde}\\
        \omega_p=\sqrt{\frac{ne^2}{\varepsilon_0m_e}} & \text{pulsation plasma}
    \end{dcases}
\quad\text{avec}\quad
    \begin{dcases}
        e=\SI{1,60e-19}{C}\\
        \varepsilon_0=\SI{8,85e-12}{F.m^{-1}}\\
        m_e=\SI{9,11e-31}{kg}\\
        n=\SI{1,00e11}{m^{-3}}\\
        c=\SI{3,00e8}{m.s^{-1}}
    \end{dcases}
\end{align*}
%-------------------------------------------------------------------------------
%-------------------------------------------------------------------------------
\begin{Exercise}
{\it Calculez la fréquence minimale avec laquelle communiquer depuis la surface avec un satellite. Pour cela, il faut que le nombre d'onde soit réel.
}
\end{Exercise}
%-------------------------------------------------------------------------------
\begin{Answer}
On veut $k^2\geqslant 0$ et donc $\omega_\text{min}=\omega_p$. D'où 
\begin{lstlisting}
e = 1.6e-19
eps0 = 8.85e-12
me = 9.11e-31
n = 1e11
c = 3e8

wp2 = n*e**2/(eps0*me)
fmin = wp2**0.5/(2*math.pi)
print(fmin)
\end{lstlisting}

On trouve $\nu=\SI{2,83e6}{Hz}$.
\end{Answer} 
%-------------------------------------------------------------------------------
%-------------------------------------------------------------------------------
\subsection{Méthode de Bessel}
%-------------------------------------------------------------------------------
%-------------------------------------------------------------------------------
Dans un home-cinéma, on veut projeter l'image d'un objet (le film) sur un écran (au mur) par une lentille. La distance objet-écran est imposée par la disposition de la salle et vaut $D=\SI{5}{m}$. La lentille de projection a pour distance focale $f'=\SI{0.4}{m}$.

Il existe deux positions de la lentille entre l'objet et l'écran telles que l'image est nette. Le grandissement (rapport entre la taille de l'objet et celle de l'image) vaut :
	\begin{align*}
	\gamma_1=\frac{D+\sqrt{D^2-4Df'}}{-D+\sqrt{D^2-4Df'}}
	\quad\text{et}\quad
	\gamma_2=\frac{D-\sqrt{D^2-4Df'}}{-D-\sqrt{D^2-4Df'}}
	\end{align*}
%-------------------------------------------------------------------------------
%-------------------------------------------------------------------------------
\begin{Exercise}
{\it Calculez les deux grandissements. Vérification : leur produit doit valoir 1.}
\end{Exercise}
%-------------------------------------------------------------------------------
\begin{Answer}
Pour minimiser les répétitions du même calcul (la racine du discriminant) on crée une variable intermédiaire.
\begin{lstlisting}
D = 5
fp = 0.4
sq_delta = (D**2-4*D*fp)**0.5

gamma1 = (D+sq_delta)/(-D+sq_delta)
gamma2 = (D-sq_delta)/(-D-sq_delta)
print(gamma1)
print(gamma2)
print(gamma1*gamma2)
\end{lstlisting}

On trouve $\gamma_1=\num{-10,4}$ et $\gamma_2=\num{-0,09}$.
\end{Answer} 
%-------------------------------------------------------------------------------
%-------------------------------------------------------------------------------
\subsection{Résonance en transmission}
%-------------------------------------------------------------------------------
%-------------------------------------------------------------------------------
Un électron d'énergie $E$ arrive sur une zone de largeur $l$ où son énergie potentielle vaut $V_0<E$. D'après les lois de la physique quantique, la probabilité qu'il reparte en arrière (réflexion) est :\footnote{Ce n'est pas demandé, mais le tracé de cette fonction $R(l)$ montre qu'il s'agit d'une fonction présentant des minima très piqués. Cela signifie que la probabilité complémentaire (de traverser la zone) $T=1-R$ présente un phénomène de \emph{résonance} (passage par un maximum) pour certaines valeurs de $l$.}
\begin{align*}
    R=\frac{(k_2^2-k_1^2)^2\sin(k_2l)^2}{4k_1^2k_2^2+(k_1^2-k_2^2)^2\sin(k_2l)^2}
    \quad\text{où}\quad
    \begin{dcases}
        k_1=\sqrt{\frac{2mE}{\hslash^2}}\\
        k_2=\sqrt{\frac{2m(E-V_0)}{\hslash^2}}
    \end{dcases}
    \quad\text{avec}\quad
    \begin{dcases}
        m=\SI{9,11e-31}{kg}\\
        E=\SI{5,00}{eV}\\
        V_0=\SI{3,00}{eV}\\
        l=\SI{5,00e-9}{m}\\
        \hslash=\SI{1,05e-34}{J.s}
    \end{dcases}
\end{align*}

Dans $k_1$ et $k_2$, il faut exprimer les énergies $E$ et $V_0$ en joules. Le facteur de conversion des électron-volt aux joules est $\SI{1}{eV}=\SI{1,60e-19}{J}$.
%-------------------------------------------------------------------------------
%-------------------------------------------------------------------------------
\begin{Exercise}
{\it Calculez numériquement la probabilité $R$.}
\end{Exercise}
%-------------------------------------------------------------------------------
\begin{Answer}
\begin{lstlisting}
m = 9.11e-31
hbar = 1.05e-34
l = 5e-9
conv_eV = 1.6e-19
E = 5*conv_eV
V0 = 3*conv_eV

k12 = 2*m*E/hbar**2
k22 = 2*m*(E-V0)/hbar**2
a = (k22-k12)**2*math.sin(k22**0.5*l)**2
R = a/(4*k12*k22+a)
print(R)
\end{lstlisting}
On trouve $R= 0,176$.
\newpage
\end{Answer} 
%-------------------------------------------------------------------------------
%-------------------------------------------------------------------------------
\subsection{Angle de perte d'une bobine}
%-------------------------------------------------------------------------------
%-------------------------------------------------------------------------------
Une bobine est un composant électronique caractérisé par deux coefficients réels positifs, son inductance $L$ et sa résistance $R$. On peut alors définir son impédance, paramètre complexe défini par $Z=R+iL\omega$, avec $\omega$ la pulsation du signal électrique utilisé dans le montage.

On définit alors son \emph{angle de perte} $\delta$ par :
\begin{align*}
    \tan(\delta)=\frac{\text{Re}(Z)}{\text{Im}(Z)}
    \quad\text{avec}\quad
    \begin{dcases}
        L=\SI{0,1}{H}\\
        R=\SI{10}{\ohm}\\
        \nu=\SI{50}{Hz}\\
        \omega=2\pi\nu\\
        \delta\in\left\lbrack 0,\frac{\pi}{2}\right\lbrack
    \end{dcases}
\end{align*}

La partie réelle et la partie imaginaire d'un complexe $u$ s'obtiennent respectivement par \type{u.real} et \type{u.imag}.

%-------------------------------------------------------------------------------
%-------------------------------------------------------------------------------
\begin{Exercise}
{\it Calculez l'angle de pertes en degrés.}
\end{Exercise}
%-------------------------------------------------------------------------------
\begin{Answer}
\begin{lstlisting}
L = 0.1
R = 10
nu = 50

omega = 2*math.pi*nu
Z = R+1j*L*omega
tand = Z.real/Z.imag
delta_rad = math.atan(tand)
delta_deg = delta_rad*180/math.pi
print(delta_deg)
\end{lstlisting}

On trouve $\delta=\ang{17,7;;}$. On peut aussi remarquer que c'est le complémentaire de $\arg(Z)$.
\end{Answer} 
%-------------------------------------------------------------------------------
%-------------------------------------------------------------------------------
\section{Calculs mathématiques}
%-------------------------------------------------------------------------------
%-------------------------------------------------------------------------------
\begin{Exercise}
{\it Écrire un programme qui écrit une durée exprimée en secondes dans une variable \type{t} sous la forme jours, heures, minutes, secondes.

Par exemple, pour \type{t = 325415} le programme devra afficher 

\type{325415 secondes correspondent à 3 jours 18h 23mn 35s}

\textit{Rappels :} \type{//} pour la division entière, \type{\%} pour le reste.}
\end{Exercise}
%-------------------------------------------------------------------------------
%-------------------------------------------------------------------------------
\begin{Exercise}
{\it Le polynôme $aX^2 + b X +c$ est défini par trois variables réelles (de type \type{float}) \type{a}, \type{b} et \type{c}. 

Écrire un programme qui affiche les deux racines du polynôme.

Que se passe-t-il lorsque le discriminant est négatif ?}
\end{Exercise}
%-------------------------------------------------------------------------------
%-------------------------------------------------------------------------------
\subsection{Polynômes de degré 3}
%-------------------------------------------------------------------------------
%-------------------------------------------------------------------------------
%-------------------------------------------------------------------------------
Le but de ces exercices est de déterminer les 3 racines (complexes) d'un polynôme de degré 3 : $P(X)=X^3+aX^2++bX+c$. Même dans le cas d'un polynôme à coefficients réels et dont les 3 racines sont réelles ont doit calculer des valeurs intermédiaires complexes. 

On peut définir un nombre complexe à partir de son module $r$ et de son argument $\varphi$ à l'aide de la fonction \type{cmath.rect(r, phi)} du module \type{cmath}.
\begin{lstlisting}
import cmath 

>>> cmath.rect(2, cmath.pi/3)
(1.0000000000000002+1.7320508075688772j)
\end{lstlisting}

%-------------------------------------------------------------------------------
%-------------------------------------------------------------------------------
\begin{Exercise}
{\it Étant donné un nombre (éventuellement complexe) dans une variable $u$, écrire les instructions qui permettent de calculer ses 3 racines troisièmes dans ${\mathbb C}$ dans des variables \type{a1}, \type{a2} et \type{a3}.}
\end{Exercise}
%-------------------------------------------------------------------------------
\begin{Answer}
\begin{lstlisting}
a1 = u**(1/3)
w = cmath.rect(1,2*cmath.pi/3)
a2 = a1*w
a3 = a2*w # ou a1/w
\end{lstlisting}
\end{Answer}
%-------------------------------------------------------------------------------
%-------------------------------------------------------------------------------
\medskip

Dans le cas du polynôme $P(X)=X^3+pX+q$ on démontre (exercice de mathématique) que si on pose $X = a+b$ en imposant $ab = -\frac p3$ alors $a^3+b^3 = -q$. Comme on a aussi $a^3b^3 = -\frac {p^3}{27}$ on voit que $a^3$ et $b^3$ sont les racines de $\displaystyle Q(X) = X^2 +qX-\frac {p^3}{27}$.

On obtient alors 2 racines $r_1$ et $r_2$. On démontre aussi que si

\begin{itemize}
    \item on choisit une des racines, $r_1$
    \item on attribue successivement à $a$ les 3 racines de $r_1$ 
    \item on détermine $b$ par $ab = -\frac p3$,
\end{itemize} 

on obtient alors les trois racines de $P$ en calculant $a+b$.
%-------------------------------------------------------------------------------
%-------------------------------------------------------------------------------
\begin{Exercise}
{\it Écrire les instructions qui permettent de calculer les 3 racines, \type{t1}, \type{t2} et \type{t3} du polynôme $X^3+pX+q$ à partir des valeurs des coefficients donnés dans deux variables \type{p} et \type{q}.

On supposera $p$ non nul}
\end{Exercise}
%-------------------------------------------------------------------------------
\begin{Answer}
\begin{lstlisting}
delta  = q**2 + 4*p**3/27
u = (-q + delta**(1/2))/2
a1 = u**(1/3)
w = cmath.rect(1,2*cmath.pi/3)
a2 = a1*w
a3 = a2*w 
b1 = -p/3/a1
b2 = -p/3/a2
b3 = -p/3/a3
t1 = a1 + b1
t2 = a2 + b2
t3 = a3 + b3
\end{lstlisting}
\newpage
\end{Answer}
%-------------------------------------------------------------------------------
%-------------------------------------------------------------------------------
\begin{Exercise}
{\it Dans le cas général, $P(X)=X^3+\alpha X^2+\beta X+\gamma$, montrer que $u$ est racine de $P$ si et seulement si $u+\frac \alpha 3$ est racine d'un polynôme de la forme $P_1(X)=X^3+pX+q$ dont on déterminera les coefficients $p$ et $q$ en fonction de $\alpha$, $\beta$  et $\gamma$.

En déduire le calcul des 3 racines du  polynôme $P$ à partir des variables \type{alpha}, \type{beta} et \type{gamma} dans trois variables \type{t1}, \type{t2} et \type{t3}

On pourra supposer qu'on a $p\ne 0$ ou traiter ce cas en utilisant une instruction conditionnelle.}
\end{Exercise}
%-------------------------------------------------------------------------------
\begin{Answer}


$\bigl(u+\frac \alpha 3\bigr)^3 = u^3+\alpha.u^2+\frac {\alpha^2}3u+\frac {\alpha^3}{27}$ or $u^3+\alpha.u^2 = -\beta.u-\gamma$ car $u$ est racine.

Ainsi $\bigl(u+\frac \alpha 3\bigr)^3 = \bigl(\frac {\alpha^2}3 - \beta)u+\frac {\alpha^3}{27}-\gamma
=\bigl(\frac {\alpha^2}3 - \beta)\bigl(u+\frac \alpha3\bigr)+\frac {\alpha^3}{27}-c-\frac {\alpha^3}9+\frac{\alpha \beta}3$.

$p = \beta - \frac {\alpha^2}3$ et $q = \gamma - \frac{\alpha\beta}3+\frac {2\alpha^3}{27}$.
\begin{lstlisting}
import cmath

alpha = -6
beta = 11
gamma = -6

p = beta - alpha**2/3
q = gamma - alpha*beta/3 + 2*alpha**3/27
d = -alpha/3
if p == 0:
    u = -q
    t1 = u**(1/3) + d
    w = cmath.rect(1,2*cmath.pi/3)
    t2 = t1*w + d
    t3 = t2*w  + d
else:
    delta  = q**2 + 4*p**3/27
    u = (-q + delta**(1/2))/2
    a1 = u**(1/3)
    w = cmath.rect(1, 2*cmath.pi/3)
    a2 = a1*w
    a3 = a2*w 
    b1 = -p/3/a1
    b2 = -p/3/a2
    b3 = -p/3/a3
    t1 = a1 + b1 + d 
    t2 = a2 + b2 + d 
    t3 = a3 + b3 + d 
print(t1, t2, t3)
\end{lstlisting}
\newpage
\end{Answer}
%-------------------------------------------------------------------------------
%-------------------------------------------------------------------------------

