%-------------------------------------------------------------------------------
%-------------------------------------------------------------------------------
%-------------------------------------------------------------------------------
\chapter{Exploitation statistique de données}
%-------------------------------------------------------------------------------
%-------------------------------------------------------------------------------
\thispagestyle{empty}
{\sf Le but de ce T.P. est de d'utiliser les instructions d'agrégation du langage SQL lors de l'exploitation d'une base de données.}
%--------------------------------------------------------------------------
%--------------------------------------------------------------------------
%--------------------------------------------------------------------------
\section{Présentation des regroupements}
%--------------------------------------------------------------------------
%--------------------------------------------------------------------------
%--------------------------------------------------------------------------
\begin{minipage}[t]{0.45\linewidth}
On prend l'exemple d'une base de notes de colles, courte pour les besoins de la présentation.

\begin{center}
\begin{tabular}[t]{|l|l|l|}
\hline
\bf Colleur  & \bf Etudiant & \bf Note \\ 
\hline
Lesec&Riri&8\\
Legaga&Riri&18\\
Lemoyen&Loulou&15\\
Legaga&Fifi&17\\
Legaga&Loulou&20\\
Lesec&Fifi&13\\
Lemoyen&Riri&15\\
Lemoyen&Fifi&15\\
Lesec&Loulou&15\\
\hline
\end{tabular}
\end{center}
\end{minipage}
%--------------------------------------------------------------------------
\hfill
%--------------------------------------------------------------------------
\begin{minipage}[t]{0.45\linewidth}
Si on veut connaître par exemple les moyennes de chaque étudiant on doit regrouper les lignes par nom d'étudiant.

L'instruction {\bf GROUP BY Etudiant} effectue ce travail virtuellement.

\begin{center}
\begin{tabular}[t]{|l|l|l|}
\hline
\bf Colleur  & \bf Étudiant & \bf Note \\ 
\hline
Legaga&Fifi&17\\
Lesec&Fifi&13\\
Lemoyen&Fifi&15\\
\hline
Lesec&Loulou&15\\
Lemoyen&Loulou&15\\
Legaga&Loulou&20\\
\hline
Lesec&Riri&8\\
Legaga&Riri&18\\
Lemoyen&Riri&15\\
\hline
\end{tabular}
\end{center}
\end{minipage}
%--------------------------------------------------------------------------

\vfill

\begin{itemize}
\item
%--------------------------------------------------------------------------
On peut alors demander les moyennes ; elles seront calculées pour chaque partie obtenue par le regroupement.

\begin{minipage}[t]{0.60\linewidth}
\begin{lstlisting}[language=SQL]
select AVG(Note)
from colles
group by Etudiant
\end{lstlisting}
\end{minipage}
%--------------------------------------------------------------------------
\begin{minipage}[t]{0.30\linewidth}
\begin{center}
\begin{tabular}[t]{|l|}
\hline
\bf AVG(Note) \\ 
\hline
15\\
16.666666667\\
13.666666667\\
\hline
\end{tabular}
\end{center}
\end{minipage}
%--------------------------------------------------------------------------

\newpage
\item
%--------------------------------------------------------------------------
On voit qu'il manque l'information de l'étudiant : {\bf quand on fait un regroupement, on doit toujours faire afficher la colonne qui a servi de critère de regroupement.}

Il est recommandé de donner un nom au calcul

\begin{minipage}[t]{0.6\linewidth}
\begin{lstlisting}[language=SQL]
select Etudiant, AVG(Note) as moy
from colles
group by Etudiant
\end{lstlisting}
\end{minipage}
%--------------------------------------------------------------------------
\begin{minipage}[t]{0.35\linewidth}
\begin{center}
\begin{tabular}[t]{|l|l|}
\hline
\bf Etudiant &\bf moy \\ 
\hline
Fifi&15\\
Loulou&16.666666667\\
Riri&13.666666667\\
\hline
\end{tabular}
\end{center}
\end{minipage}
%--------------------------------------------------------------------------

\vskip 1cm
\item
Il est par contre totalment inutile d'afficher d'autres colonnes que la colonne de critère et les valeur calculées, le logiciel donnera une réponse mais celle-ci n'a aucun sens.

\begin{lstlisting}[language=SQL]
select Etudiant, AVG(Note) as moy, Colleur
from colles
group by Etudiant
\end{lstlisting}

\begin{center}
\begin{tabular}[t]{|l|l|l|}
\hline
\bf Etudiant &\bf  my &\bf  Colleur \\ 
\hline
Fifi&15&Legaga\\
Loulou&16.666666667&Lesec\\
Riri&13.666666667&Lesec\\
\hline
\end{tabular}
\end{center}
%--------------------------------------------------------------------------

\vskip 1cm
\item 
%--------------------------------------------------------------------------
Si on veut sélectionner selon des valeurs calculées, on ne peut pas le faire avec \type{where} car le critère de cet instruction a été utilisé {\bf avant} le regroupement. L'instruction {\bf HAVING}, placée après \type{group by} permet d'appliquer le critère. On voit ici l'utilité de nommer le calcul.

\begin{minipage}[t]{0.60\linewidth}
\begin{lstlisting}[language=SQL]
select Etudiant, AVG(Note) as moy
from colles
group by Etudiant
having moy > 14
\end{lstlisting}
\end{minipage}
%--------------------------------------------------------------------------
\begin{minipage}[t]{0.35\linewidth}
\begin{center}
\begin{tabular}[t]{|l|l|}
\hline
\bf Etudiant &\bf moy \\ 
\hline
Fifi&15\\
Loulou&16.666666667\\
\hline
\end{tabular}
\end{center}
\end{minipage}
%--------------------------------------------------------------------------
\end{itemize}
%--------------------------------------------------------------------------
%--------------------------------------------------------------------------
\newpage
%--------------------------------------------------------------------------
\section{Requêtes}
%--------------------------------------------------------------------------
%--------------------------------------------------------------------------
%--------------------------------------------------------------------------
\subsection{Regroupements}
%--------------------------------------------------------------------------
%--------------------------------------------------------------------------
\begin{Exercise}
\it Calculer le nombre de communes par département.\\
Quels sont les 3 départements qui comportent le plus grand nombre de communes ? (62, 2, 80)
\end{Exercise}
%--------------------------------------------------------------------------
\begin{Answer}
\begin{lstlisting}[language=SQL]
select dept, count() as nombreCommunes
from communes
group by dept
order by 2 desc
limit 3
\end{lstlisting}
\end{Answer}
%--------------------------------------------------------------------------
%--------------------------------------------------------------------------
\begin{Exercise}
\it Quels sont les 3 départements les plus peuplés ?
(59, 75, 13)
\end{Exercise}
%--------------------------------------------------------------------------
\begin{Answer}
\begin{lstlisting}[language=SQL]
select dept, sum(POP2010) as pop
from communes
group by dept
order by 2 desc
limit 3
\end{lstlisting}
\end{Answer}
%--------------------------------------------------------------------------
%--------------------------------------------------------------------------
\begin{Exercise}
\it Déterminer la densité de population par département.
\end{Exercise}
%--------------------------------------------------------------------------
\begin{Answer}
\begin{lstlisting}[language=SQL]
select dept, sum(POP2010)/sum(surf) as densité
from communes
group by dept
\end{lstlisting}
\end{Answer}
%--------------------------------------------------------------------------
%--------------------------------------------------------------------------
\begin{Exercise}
\it Quels est le nombre de communes par arrondissement ?
\end{Exercise}
%--------------------------------------------------------------------------
\begin{Answer}
\begin{lstlisting}[language=SQL]
select arr, count()
from communes
group by arr
\end{lstlisting}
\end{Answer}
%--------------------------------------------------------------------------
%--------------------------------------------------------------------------
\begin{Exercise}[label = ques:nomPlus]
\it Quels est le nom de commune le plus courant ?

14 communes s'appellent Sainte Colombe.
\end{Exercise}
%--------------------------------------------------------------------------
\begin{Answer}
\begin{lstlisting}[language=SQL]
select nom, count() as nombre
from communes
group by nom
order by 2 desc
limit 1
\end{lstlisting}
\end{Answer}
%--------------------------------------------------------------------------
%--------------------------------------------------------------------------
\begin{Exercise}[label = ques:nomPlus]
\it Quels est le nombre de départements par région ?

En 2010, il y avait 21 régions dont la table ne donne que le numéro.

Pour compter les départements sans doublon \type{COUNT(dept)} ne suffit pas, il faut écrire \type{COUNT(DISTINCT dept)}.
\end{Exercise}
%--------------------------------------------------------------------------
\begin{Answer}
\begin{lstlisting}[language=SQL]
select region, count(distinct dept) as nb
from communes
group by region
\end{lstlisting}
\end{Answer}
%--------------------------------------------------------------------------
%--------------------------------------------------------------------------
\subsection{Utilisation de \type{HAVING}}
%--------------------------------------------------------------------------
%--------------------------------------------------------------------------
\begin{Exercise}
\it Quels sont les département de population supérieure à 1.500.000 ? (59, 75, 13, 69, 92, 93).
\end{Exercise}
%--------------------------------------------------------------------------
\begin{Answer}
\begin{lstlisting}[language=SQL]
select dept, sum(POP2010) as population
from communes
group by dept
having population > 1 500 000
order by 2 desc
\end{lstlisting}
\newpage
\end{Answer}
%--------------------------------------------------------------------------
%--------------------------------------------------------------------------
\begin{Exercise}
\it Quels sont les départements dont la population de la ville la plus peuplée est au moins un tiers de la population totale ? (13, 31, 75, 87, 90).
\end{Exercise}
%--------------------------------------------------------------------------
\begin{Answer}
\begin{lstlisting}[language=SQL]
select dept, max(pop2010) as popMax, sum(pop2010) as pop
from communes
group by dept
having pop < 3*popMax
\end{lstlisting}
\end{Answer}
%--------------------------------------------------------------------------
%--------------------------------------------------------------------------
\begin{Exercise}
\it Quels sont les arrondissements qui ne contiennent qu'une commune ? On donnera aussi le département.
\end{Exercise}
%--------------------------------------------------------------------------
\begin{Answer}
\begin{lstlisting}[language=SQL]
select dept, arr, count() as nb
from communes
group by arr
having nb = 1
\end{lstlisting}
\end{Answer}
%--------------------------------------------------------------------------
%--------------------------------------------------------------------------
\begin{Exercise}
\it Quels sont les départements dont la population de plus de 60 ans représente plus de 15\% de la population totale ? (10 départements).
\end{Exercise}
%--------------------------------------------------------------------------
\begin{Answer}
\begin{lstlisting}[language=SQL]
select dept, 100*(sum(age60_74)+sum(age75+0.0))/sum(pop2010) as taux_vieux
from communes
group by dept
having taux_vieux>30
\end{lstlisting}
\end{Answer}
%--------------------------------------------------------------------------
%--------------------------------------------------------------------------
\begin{Exercise}
\it Quels sont les départements dont la population a crû de plus de 15\% entre 1999 et 2010 (il y en a 8) ?
\end{Exercise}
%--------------------------------------------------------------------------
\begin{Answer}
\begin{lstlisting}[language=SQL]
select dept, (sum(pop2010) - sum(pop1999) + 0.0)/sum(pop1999) as taux
from communes
group by dept
having taux > 0.15
\end{lstlisting}
\end{Answer}
%--------------------------------------------------------------------------
\newpage
%--------------------------------------------------------------------------
%--------------------------------------------------------------------------
\section{Requêtes imbriquées}
%--------------------------------------------------------------------------
%--------------------------------------------------------------------------
%--------------------------------------------------------------------------
Si le résultat d'une requête est un tableau on peut l'utiliser pour une autre requête.

Par exemple après avoir calculé la population par arrondissement,
\begin{lstlisting}[language=SQL]
select dept, arr, sum(POP2010 )as s
from communes
group by arr
\end{lstlisting}

on peut calculer la moyenne des habitants par arrondissement pour chaque département.

\begin{lstlisting}[language=SQL]
select dept, avg(s) as moyenne
from (select dept, arr, sum(POP2010 )as s
      from communes group by arr)
group by dept
order by moyenne desc
\end{lstlisting}


Si une requête produit un nombre (\type{count}, \type{avg}, \type{max}, \dots) on peut l'utiliser dans un test ou un calcul.
%--------------------------------------------------------------------------
%--------------------------------------------------------------------------
\begin{Exercise}
\it Après avoir calculé la population par département, déterminer la moyenne de la population par département (664\,420,7).
\end{Exercise}
%--------------------------------------------------------------------------
\begin{Answer}
\begin{lstlisting}[language=SQL]
select  dept, sum(POP2010 )as population
from communes
group by dept
\end{lstlisting}
\begin{lstlisting}[language=SQL]
select round(avg(population),1)
from (select  dept, sum(POP2010 ) as population
      from communes
      group by dept)
\end{lstlisting}
\end{Answer}
%--------------------------------------------------------------------------
%--------------------------------------------------------------------------
\begin{Exercise}
\it Déterminer les départements qui ont une population inférieure au quart de la moyenne de la population par département (48, 23, 5, 90, 15, 9, 4).
\end{Exercise}
%--------------------------------------------------------------------------
\begin{Answer}
\begin{lstlisting}[language=SQL]
select dept, population
from (select  dept, sum(POP2010 ) as population 
      from communes 
      group by dept)
where population <  (select  avg(pop) 
                     from (select  dept, sum(POP2010 ) as pop
                           from communes
                           group by dept))/4.0
order by 2 desc
\end{lstlisting}
\newpage
\end{Answer}
%--------------------------------------------------------------------------
%--------------------------------------------------------------------------
\begin{Exercise}
\it Déterminer le nombre d'arrondissements par département.
\end{Exercise}
%--------------------------------------------------------------------------
\begin{Answer}
\begin{lstlisting}[language=SQL]
select dept, count() as nbArr
from (select dept, arr,sum(pop2010) as popArr
      from communes
      group by arr)
group by dept
\end{lstlisting}
\end{Answer}
%--------------------------------------------------------------------------
%--------------------------------------------------------------------------
\begin{Exercise}
\it Déterminer le nombre de département qui ont 1, 2, 3, \dots, 9 arrondissements.
\end{Exercise}
%--------------------------------------------------------------------------
\begin{Answer}
\begin{lstlisting}[language=SQL]
select nbArr, count() as nb
from (select dept, count() as nbArr
      from (select dept, arr,sum(pop2010) as popArr
            from communes
            group by arr)
      group by dept)
group by nbArr
\end{lstlisting}
\end{Answer}
%--------------------------------------------------------------------------
%--------------------------------------------------------------------------
\begin{Exercise}
\it Quels sont les département dans lesquels sont situés les communes ayant le nom le plus courant (question \ref{ques:nomPlus}) ?.

On ne devra pas écrire "Sainte Colombe".
\end{Exercise}
%--------------------------------------------------------------------------
\begin{Answer}
\begin{lstlisting}[language=SQL]
select dept,  nom
from communes
where nom =(select nom  
            from (select nom, count() as nombre
                  from communes
                  group by nom
                  order by 2 desc
                  limit 1
                 )
           )
\end{lstlisting}
\end{Answer}
%--------------------------------------------------------------------------
%--------------------------------------------------------------------------
\begin{Exercise}
\it Déterminer le département dont la surface est la plus proche de la surface moyenne.
\end{Exercise}
%--------------------------------------------------------------------------
\begin{Answer}
\begin{lstlisting}[language=SQL]
select dept, surface
from (select dept, sum(surf) as surface 
      from communes group by dept)
order by abs(surface - (select avg(surface) 
                        from (select dept, sum(surf) as surface 
                              from communes group by dept
                             )
                       )
            )
limit 1		     
\end{lstlisting}
\end{Answer}
%--------------------------------------------------------------------------
%--------------------------------------------------------------------------
On pourra trier selon la valeur absolue de la différence de la surface avec la moyenne et ne garder qu'un résultat ({limit 1}).
Le département est 15.
%--------------------------------------------------------------------------

\medskip
%--------------------------------------------------------------------------
On veut classer les villes par leur taille en les regroupant sous la forme 1 à 9 habitants, 10 à 99 habitants, etc
Pour cela la fonction \type{length(n)}, qui renvoie le nombre de chiffres de $n$, est utile.
%--------------------------------------------------------------------------
%--------------------------------------------------------------------------
\begin{Exercise}
\it Déterminer le nombre de personnes habitant dans une ville comportant 1 à 9 habitants, 10 à 99 habitants, \dots
\end{Exercise}
%--------------------------------------------------------------------------
\begin{Answer}
\begin{lstlisting}[language=SQL]
select sum(POP2010) as population, count() as nombre,  rang
from (select nom, dept, POP2010, length(POP2010) as rang
      from communes)
group by rang
\end{lstlisting}
\newpage
\end{Answer}
%--------------------------------------------------------------------------
%--------------------------------------------------------------------------
\subsection{Hors-programme : utilisation de \type{IN}}
%--------------------------------------------------------------------------
%--------------------------------------------------------------------------
On peut aussi utiliser une table calculée dans une condition \type{where} : si on produit un ensemble de valeurs on peut tester l'appartenance d'une valeur (ou d'un tuple de valeurs) à cette table avec le test \type{x IN table}.
%--------------------------------------------------------------------------
%--------------------------------------------------------------------------
\begin{Exercise}
\it Donner le nom de la ville la plus peuplée par département.
\end{Exercise}
%--------------------------------------------------------------------------
\begin{Answer}
\begin{lstlisting}[language=SQL]
select dept, nom, pop2010
from communes
where (dept, pop2010) in (select dept, max(pop2010)
                      from communes
					  group by dept)
\end{lstlisting}
\end{Answer}
%--------------------------------------------------------------------------
%--------------------------------------------------------------------------




























