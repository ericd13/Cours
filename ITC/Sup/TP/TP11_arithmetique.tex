%-------------------------------------------------------------------------------
%-------------------------------------------------------------------------------
%-------------------------------------------------------------------------------
\chapter{Arithmetique}
%-------------------------------------------------------------------------------
%-------------------------------------------------------------------------------
\thispagestyle{empty}
%-------------------------------------------------------------------------------
%-------------------------------------------------------------------------------
\vskip -2cm

{\sf Dans ce T.P. nous allons utiliser les outils vus jusqu'à présent pour étudier quelques constructions arithmétiques. On rappelle que $k$ divise $n$ si et seulement si \type{n\%k} vaut 0 (pour $k> 0$).
}
%-------------------------------------------------------------------------------
%-------------------------------------------------------------------------------
%-------------------------------------------------------------------------------
\section{Nombres premiers} 
%-------------------------------------------------------------------------------
%-------------------------------------------------------------------------------
%-------------------------------------------------------------------------------
\begin{Exercise}[title = Diviseurs]
\it Écrire une fonction \type{diviseurs(n)} qui renvoie la liste de tous les diviseurs de $n$, 1 et $n$ compris. On supposera que $n$ un entier strictement positif.

\type{diviseurs(30)} renvoie \type{1, 2, 3, 5, 6, 10, 15, 30]}
\end{Exercise}
%-------------------------------------------------------------------------------
\begin{Answer}
\begin{lstlisting}
def diviseurs(n):
    div = []
    for k in range(1, n+1):
        if n%k == 0:
            div.append(k)
    return div
\end{lstlisting}

On teste les divisibilités pour les entiers supérieurs à $\frac n2$, ce qui est inutile.

\begin{lstlisting}
def diviseurs(n):
    div = []
    for k in range(1, n//2+1):
        if n%k == 0:
            div.append(k)
    div.append(n)
    return div
\end{lstlisting}

\end{Answer}
%-------------------------------------------------------------------------------
%-------------------------------------------------------------------------------
On rappelle qu'un entier strictement positif est {\bf premier} si et seulement si il admet exactement deux diviseurs, 1 et lui-même, ce qui exclut 1 des nombres premiers.
%-------------------------------------------------------------------------------
%-------------------------------------------------------------------------------
\begin{Exercise}[title = Test de primalité 1]
\it En utilisant la fonction précédente écrire une fonction \type{premier(n)} qui renvoie \type{True} ou \type{False} selon que $n$ est premier ou non.
\end{Exercise}
%-------------------------------------------------------------------------------
\begin{Answer}
\begin{lstlisting}
def premier(n):
    return len(diviseurs(n)) == 2
\end{lstlisting}
\end{Answer}
%-------------------------------------------------------------------------------
%-------------------------------------------------------------------------------
On sait aussi que, si $n$ n'est pas premier, il admet un diviseur $m$ avec $2 \le m\le \sqrt n$.
%-------------------------------------------------------------------------------
%-------------------------------------------------------------------------------
\begin{Exercise}[title = Test de primalité 2]
\it En utilisant cette propriété écrire une fonction \type{premier(n)} qui renvoie \type{True} ou \type{False} selon que $n$ est premier ou non. Cette fonction ne devra pas faire plus que $\sqrt n$ tests de divisibilité.
\end{Exercise}
%-------------------------------------------------------------------------------
\begin{Answer}
\begin{lstlisting}
def premier(n):
    p = 2
    while p*p <= n:
        if n%p == 0:
            return False
        p = p + 1
    return p > 1 # 1 n'est pas premier
\end{lstlisting}
\end{Answer}
%-------------------------------------------------------------------------------
%-------------------------------------------------------------------------------
Il est parfois utile de disposer de la liste des premiers nombres premiers.

Le crible d'Erathostène (vers -220 avant J.C.) permet de le faire de manière efficace. La description qui suit est celle de Wikipedia.

L'algorithme procède par élimination : il s'agit de supprimer d'une table des entiers de 2 à $N$ tous les multiples d'un entier (autres que lui-même). En supprimant tous ces multiples, à la fin il ne restera que les entiers qui ne sont multiples d'aucun entier à part 1 et eux-mêmes, et qui sont donc les nombres premiers.

\begin{enumerate}
    \item On commence par rayer les multiples de 2, puis les multiples de 3 restants, puis de 5,  et ainsi de suite en rayant à chaque fois tous les multiples du plus petit entier restant.
    \item On peut s'arrêter lorsque le carré du plus petit entier restant est supérieur au plus grand entier restant, car dans ce cas, tous les non-premiers ont déjà été rayés précédemment.
    \item À la fin du processus, tous les entiers qui n'ont pas été rayés sont les nombres premiers inférieurs à $N$.
\end{enumerate}
On pourra utiliser une liste de booléens pour implémenter la table des entiers à rayer.
%-------------------------------------------------------------------------------
%-------------------------------------------------------------------------------
\begin{Exercise}[title = Crible]
\it Écrire une fonction \type{crible(n)} qui renvoie la liste des entiers premiers de 1 à  $n$ en utilisant cette méthode.

\type{crible(30)} renvoie \type{[2, 3, 5, 7, 11, 13, 17, 19, 23, 29]}
\end{Exercise}
%-------------------------------------------------------------------------------
\begin{Answer}
\begin{lstlisting}
def crible(n):
    era = [True]*(n+1)
    prem = []
    era[0] = False
    era[1] = False
    for k in range(2, n+1):
        if era[k]:
            prem.append(k)
            for i in range(2*k, n+1, k):
                era[i] = False
    return prem
\end{lstlisting}
\newpage
\end{Answer}
%-------------------------------------------------------------------------------
%-------------------------------------------------------------------------------
%-------------------------------------------------------------------------------
\section{P.G.C.D} 
%-------------------------------------------------------------------------------
%-------------------------------------------------------------------------------
%-------------------------------------------------------------------------------
Le {\bf PGCD} de deux entier $a$ et $b$ est le plus grand entier, noté $a\wedge b$ qui divise $a$ et $b$.

Par exemple $12 \wedge 30 = 6$, $25\wedge 33 = 1$, $91\wedge 169 = 13$.

le TP5 proposait d'écrire une fonction simple
\begin{lstlisting}
def pgcd(a,b):
    commun = 1
    for k in range(1, min(a,b)+1):
        if (a % k == 0) and (b % k == 0 ):
            commun = k
    return commun
\end{lstlisting}
On peut prouver que, si $b = 0$, $a\wedge b=a$ et si $b\ne 0$ et $a = b.d + r$ est la division euclidienne de $a$ par $b$ (\type{d = a//b} et \type{r = a\%b}) alors $a\wedge b = b\wedge r$. 

C'est l'algorithme d'Euclide (vers -300 avant J.C.)
%-------------------------------------------------------------------------------
%-------------------------------------------------------------------------------
\begin{Exercise}[title = Diviseurs]
\it Écrire une fonction \type{pgcd(a, b)} qui calcule le PGCD de $a$ et $b$ en utilisant l'algorithme d'Euclide.
\end{Exercise}
%-------------------------------------------------------------------------------
\begin{Answer}
\begin{lstlisting}
def pgcd(a, b):
    while b > 0:
        r = a%b
        a = b
        b = r
    return a
\end{lstlisting}
\end{Answer}
%-------------------------------------------------------------------------------
%-------------------------------------------------------------------------------

Le théorème de Bezout énonce qu'il existe des entiers $p$ et $q$ tels que $a.p + b.q = a\wedge b$. Il n'y a pas unicité des coefficients $p$ et $q$ ; par exemple

$13 = 91\wedge 169 = 2\times 91 - 169 = -11\times 91 + 6\times 169 = 15\times91 - 8\times 169 = \cdots$

On remarque que $p$ ou $q$ peut être négatif.

Pour calculer les coefficients $p$ et $q$ on propose l'algorithme suivant.
\begin{itemize}
    \item Si $a^b=b$, on choisit $p = 0$ et $q = 1$.
    \item Si $r = a\wedge b < b$, on cherche le premier entier positif $p$ tel que \type{p*a \% b == r} (on admet qu'il existe) et on pose alors \type{q = -((p*a)//b)}. 
\end{itemize}
%-------------------------------------------------------------------------------
%-------------------------------------------------------------------------------
\begin{Exercise}[title = Coefficients de Bezout]
\it Écrire une fonction \type{bezout(a, b)} qui calcule un couple d'entier \type{p, q)} tels que $a.p + b.q = a\wedge b$.
\end{Exercise}
%-------------------------------------------------------------------------------
\begin{Answer}
\begin{lstlisting}
def bezout(a, b):
    r = pgcd(a, b)
    if r == b:
        return (0, 1)
    else:
        p = 1
        while p*a % b != r:
            p = p + 1
        return (p, -((p*a)//b))
\end{lstlisting}
\end{Answer}
%-------------------------------------------------------------------------------
%-------------------------------------------------------------------------------
%-------------------------------------------------------------------------------
\section{Compléments (plus difficiles)} 
%-------------------------------------------------------------------------------
%-------------------------------------------------------------------------------
%-------------------------------------------------------------------------------
\begin{Exercise}[title = Diviseurs]
\it Écrire une fonction \type{decomposition(n)} qui calcule la liste des diviseurs premiers de $n$ avec répétition et par ordre croissant.

\type{decomposition(180)} renvoie \type{[2, 2, 3, 3, 5]}
\end{Exercise}
%-------------------------------------------------------------------------------
\begin{Answer}
\begin{lstlisting}
def decomposition(n):
    facteurs = []
    d = 2
    while d <= n:
        if n%d == 0:
            facteurs.append(d)
            n = n//d 
        else:
            d = d + 1
    return facteurs
\end{lstlisting}
\end{Answer}
%-------------------------------------------------------------------------------
%-------------------------------------------------------------------------------
\begin{Exercise}[title = Intersection]
\it Écrire une fonction \type{inter(L1, L2)} qui calcule la liste des éléments communs aux deux listes \type{L1} et \type{L2} avec répétitions. On écrira un algorithme rapide qui utilise le fait que  \type{L1} et \type{L2} sont {\bf croissantes}.

\type{inter([2, 5, 5, 7, 11], [2, 3, 3, 5, 5, 5])} renvoie \type{[2, 5, 5]}
\end{Exercise}
%-------------------------------------------------------------------------------
\begin{Answer}
\begin{lstlisting}
def inter(L1, L2):
    n1 = len(L1)
    n2 = len(L2)
    int = []
    i1 = 0
    i2 = 0
    while i1 < n1 and i2 < n2:
        if L1[i1] == L2[i2]:
            int.append(L1[i1])
            i1 = i1 + 1
            i2 = i2 + 1
        elif L1[i1] < L2[i2]:
            i1 = i1 + 1
        else:
            i2 = i2 + 1
    return int
\end{lstlisting}
\newpage
\end{Answer}
%-------------------------------------------------------------------------------
%-------------------------------------------------------------------------------
\begin{Exercise}[title = PGCD avec les décompositions]
\it En déduire une fonction \type{pgcd\_dec(a, b)} qui calcule le PGCD en utilisant que sa décomposition est l'intersection des décompositions en facteurs premiers de $a$ et de $b$.
\end{Exercise}
%-------------------------------------------------------------------------------
\begin{Answer}
\begin{lstlisting}
def produit(liste):
    p = 1
    for x in liste:
        p = p*x
    return p
\end{lstlisting}

\begin{lstlisting}
def pgcd_dec(a, b):
    d1 = decomposition(a)
    d2 = decomposition(b)
    return produit(inter(d1, d2))
\end{lstlisting}
\end{Answer}
%-------------------------------------------------------------------------------
%-------------------------------------------------------------------------------
\begin{Exercise}[title =Coefficients de Bezout bis]
\it Écrire une fonction \type{bezout(a, b)} qui calcule un couple d'entier \type{p, q)} tels que $a.p + b.q = a\wedge b$. On donnera un algorithme plus rapide que précédemment en suivant l'algorithme d'Euclide.
\end{Exercise}
%-------------------------------------------------------------------------------
\begin{Answer}
On note $a_k$ les valeurs successives de $a$.

On a $a_0 = a$ et $a_1 = b$, puis $a_{k+2} = a_k \% a_{k+1}$.

On a $a_0 = a = 1.a + 0.b$ et $a_1 = b = 0.a + 1.b$. On cherche $p_k$ et $q_k$ tels que $a_k = p_k.a + q_k.b$. 

L'égalité $a_{k+2} = a_k \% a_{k+1} = a_k - (a_k//a_{k+1})*a_{k+1}$ donne $p_{k+2} = p_k - (a_k//a_{k+1})*p_{k+1}$ et $q_{k+2} = q_k - (a_k//a_{k+1})*q_{k+1}$
\begin{lstlisting}
def bezout(a, b):
    p_a = 1
    q_a = 0
    p_b = 0
    q_b = 1
    while b > 0:
        r = a%b
        d = a//b
        a = b
        b = r
        p = p_a - d*p_b
        q = q_a - d*q_b
        p_a = p_b
        q_a = q_b
        p_b = p
        q_b = q
    return a, p_a, q_a
\end{lstlisting}
\end{Answer}
%-------------------------------------------------------------------------------
%-------------------------------------------------------------------------------


